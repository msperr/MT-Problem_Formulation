\section{Problem Formulation}

\subsection{Notation and Model}

This formulation models the problem of optimal integration of autonomous vehicles in car sharing, considering multimodal transport.

\paragraph{Notation} \parfill

We are given a set of vehicles $\mathcal{V}$ and a set of costumers $\mathcal{C}$. For public transport, we have a set of available stations $\mathcal{S}$ and a set of public transport rides $\mathcal{P}$. A ride $p\in \mathcal{P}$ is a sequence of stops at time points $p=\left(\left(s_1,z_1\right),\dots,\left(s_k,z_k\right)\right)$ with $s_i\in\mathcal{S}$ and $z_i$ a time point for $i\in[k]$. 

We are further given a set of trips $\mathcal{T}$; each trip $t\in\mathcal{T}$ is either a car trip or a public transport trip and has a start and end location $p_t^{\operatorname{start}}, p_t^{\operatorname{end}}$ and a start and end time $z_t^{\operatorname{start}}, z_t^{\operatorname{end}}$. Accordingly, we define $\mathcal{T} = \mathcal{T}_{\operatorname{car}}\cupdot\mathcal{T}_{\operatorname{public}}$. A public transport trip $t\in\mathcal{T}$ is a subsequence of a public transport ride $p\in\mathcal{P}$ and it holds

\begin{align*}
	p_t^{\operatorname{start}} = s^p_i & & p_t^{\operatorname{end}} = s^p_j & & z_t^{\operatorname{start}} = z^p_i & & z_t^{\operatorname{end}} = z^p_j
\end{align*}

for some $i<j$.

The start position and the starting time of a vehicle $v\in\mathcal{V}$ is $p_v$ and $z_v$. The time, a vehicle needs from its start location to a trip or from one trip to another is $t_{s,t}$ for $s\in\mathcal{V}\cup\mathcal{T}, t\in\mathcal{T}$.

We define a partial order $\preceq$ on $\mathcal{V}\times\mathcal{T}$. We say

\begin{align*}
	s\preceq t & & \text{if } z_s^{\operatorname{end}} + t_{s,t} \leq z_t^{\operatorname{start}} & & \text{for } s\in\mathcal{V}\times\mathcal{T}, t\in\mathcal{T}
\end{align*}

The expression $s\preceq t$ means, that one car is able to fulfill both trips, first $s$ and then $t$. If one of the trips is a public transport trip, one costumer can use both of these trips.

We are given a set of multimodal routes $\mathcal{M}$. A route $m=\left(t_1,\dots,t_k\right)$ is a sequence of trips with to following properties:

\begin{align*}
	p_{t_i}^{\operatorname{end}} = p_{t_{i+1}}^{\operatorname{start}} & & t_i\preceq t_{i+t} & & t_i\in\mathcal{T}_{\operatorname{car}}\Rightarrow t_{i+1}\in\mathcal{T}_{public} & & \text{for all } i\in[k-1]
\end{align*}

We define the route start and end locations and times

\begin{align*}
	p_m^{\operatorname{start}} = p_{t_1}^{\operatorname{start}} & & p_m^{\operatorname{end}} = p_{t_k}^{\operatorname{end}} & & z_m^{\operatorname{start}} = z_{t_1}^{\operatorname{start}} & & z_m^{\operatorname{end}} = z_{t_k}^{\operatorname{end}}.
\end{align*}

Each costumer $c\in\mathcal{C}$ has a set of alternative routes. $C:\mathcal{M}\to\mathcal{C}$ maps the routes to costumers and $M:\mathcal{T}\to\mathcal{M}$ maps the trips to routes. For each route of the same costumer $m\in C^{-1}(c)$, the start and end positions are the same, the start and end times may differ.

Additionally, we have a set of refuel stations $\mathcal{R}$. A refuel station $r\in\mathcal{R}$ has the location $p_r$. In this model, a car is allowed to refuel at most once between two trips. We define $f_{s,t}^{\operatorname{d}}$ for $s\in\mathcal{V}\cup\mathcal{T}\cup\mathcal{R},t\in\mathcal{T}\cup\mathcal{R}$ as the amount, the fuel level decreases along the deadhead trip. $f_t^{\operatorname{t}}$ for $t\in\mathcal{T}\cup\mathcal{R}$ is the amount of fuel, the car needs for a trip. For $r\in\mathcal{R}$ holds $f_r^{\operatorname{t}}\leq 0$. 

\paragraph{Problem Description} \parfill

A feasible solution is a schedule of trips for every vehicle including refueling stops and a sequence of trips for every costumer. These trips are fulfilled by the scheduled car or by public transport according to its timetable. For this, we have the following constraints:

\begin{itemize}
	\item{Each car is able to serve its scheduled trips, considering time and location.}
	\item{The fuel state of each car is always in a feasible range.}
	\item{Each costumer is able to complete his trip, considering time and location.}
	\item{For each costumer, exactly one trip is chosen.}
\end{itemize}

The goal is to find a cost-minimal feasible schedule considering all these constraints.

\paragraph{Costs} \parfill

We have the following types of costs:

\begin{itemize}
	\item{Vehicles costs $c^{\operatorname{v}}$: unit costs for each used car}
	\item{Deadhead costs $c_{s,t}^{\operatorname{d}}$ for $s\in\mathcal{V}\cup\mathcal{T}\cup\mathcal{R}, t\in\mathcal{T}\cup\mathcal{R}$: costs, if a car drives to a trip or a refuel station without a costumer using it}
	\item{Trip costs $c_t^{\operatorname{t}}$ for $t\in\mathcal{T}_{\operatorname{car}}$: costs for fulfilling a trip}
\end{itemize}

For public transport, we define either trip costs for each public transport trip or fixed costs for each costumer using public transport. Finally, we define costs to consider the costumer preferences. These costs can be the total time or the number of changes.

\begin{itemize}
	\item{Trip costs $c_t^{\operatorname{t}}$ for $t\in\mathcal{T}_{\operatorname{public}}$: costs for using public transport}
	\item{Route-dependent costs $c_m^{\operatorname{r}}$ for $m\in\mathcal{M}$: costs for costumer preferences and unit costs for using public transport}
\end{itemize}

%-------------------------------------------------------------------------------------------------------------------------------------

\subsection{LP Considerations}

\paragraph{Creation of Routes} \parfill

In reality, we are not given a set of multimodal routes. We have only $\mathcal{C}, \mathcal{S}, \mathcal{P}, \mathcal{V}$. For each costumer $c\in\mathcal{C}$, we have a start and end location $p_c^{\operatorname{start}}, p_c^{\operatorname{end}}$ and a time interval $\left[z_c^{\operatorname{start}},z_c^{\operatorname{end}}\right]$, in which all routes are located.

How we determine the routes, we have not yet considered. Since a car can drive to every station, where the public transport trip starts, the number of alternative routes can be very large. Therefore, we will have to develop a preprocessing in order to reduce the number of alternatives.

\paragraph{LP Constraints} \parfill

We build on the (MILP) formulation (Kaiser, Knoll, cap. 3). For our problem, we make the following adaptions: The variables $x_{s,t}\in \{0,1\}, z_{s,r,t}\in \{0,1\}, e_s \in[0,1]$ are the same as in the (MILP). For the trips in this formulation, only the car trips $t\in\mathcal{T}_{\operatorname{car}}$ are considered.

We initialize new variables $u_t \in\{0,1\}$ for $t\in\mathcal{T}_{\operatorname{public}}$ and $v_m \in\{0,1\}$ for $m\in\mathcal{M}$. $u_t$ indicate, whether a public transport trip is fulfilled or not; $v_m$ indicate, whether a multimodal route is fulfilled. 

We replace the constraint $\eqref{eq:costumer_prev}$ by

\begin{align}
	& \sum_{m\in C^{-1}(c)} v_m = 1 & & \text{for all } c\in\mathcal{C} \label{eq:costumer_new} \\
	& \sum_{s\in\Nin(t)} x_{s,t} \geq v_m & & \text{for all } m\in\mathcal{M}, t\in M^{-1}(m)\cap\mathcal{T}_{\operatorname{car}} \label{eq:route_car} \\
	& u_t \geq v_m & & \text{for all } m\in\mathcal{M}, t\in M^{-1}(m)\cap\mathcal{T}_{\operatorname{public}} \label{eq:route_public}
\end{align}

This formulation is equivalent to $\eqref{eq:costumer}$ and $\eqref{eq:route}$, but is more suitable to formulate the objective function.

\paragraph{Objective Function} \parfill

This objective function considers unit vehicle costs, unit public transport costs, trip costs for vehicles and public transport, deadhead costs for vehicles and user preferences.

\begin{align}
	\min & \sum_{s\in\mathcal{V}} \sum_{t\in\Nout(s)\backslash\{d^{\operatorname{e}}\}} x_{s,t}c^{\operatorname{v}} + \sum_{t\in\mathcal{T}_{\operatorname{public}}} u_t c_t^{\operatorname{t}} + \sum_{m\in\mathcal{M}} v_m c_m^{\operatorname{r}} \nonumber \\
	& + \sum_{t\in\mathcal{T}_{\operatorname{car}}} \sum_{s\in\Nin(t)} \left[x_{s,t}\left(c_{s,t}^{\operatorname{d}}+c_{t}^{\operatorname{t}}\right) + \sum_{r\in\Rst} z_{s,r,t}\left(c_{s,r}^{\operatorname{d}}+c_{r,t}^{\operatorname{d}}-c_{s,t}^{\operatorname{d}}\right)\right] \tag{MILP'} \label{eq:MILP_new}
\end{align}