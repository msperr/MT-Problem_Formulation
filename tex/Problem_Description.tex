\chapter{Problem Description and Classification}
\label{ch:problem_description}

In this chapter, the problem is stated in detail and the notation is introduced. The problem is classified by relating it to known problems in literature and its complexity is determined. Finally, known approaches to similar problems are regarded. Most of the following considerations are already part of the underlying theses \cite{Kaiser} and \cite{Knoll}, except for the fact that multiple legs are allowed. All crucial results are repeated here for clarity.

\section{Situation and Issue}

We regard the situation of free-floating car sharing as it exists today in combination with autonomous vehicles. Free-floating car sharing means that a customer can rent an available car wherever and whenever one is available and use it as long as he needs. After usage, he parks the car somewhere in the operation area. We assume the existence of autonomous vehicles which behave the same as if a human were driving, but without a human being necessarily present. Instead of looking for a car, a customer books a car via a smartphone application and gets picked up by the car at the desired start location at the desired start time. For the customer, this would be similar to a taxi service. 

The car sharing issue is combined with public transport as it is known today. There is a fixed schedule, according to which the bus or train visits public transport stations in a row at certain time points. A possible route for a customer may look as follows: The customer is picked up at his start position by a car and is brought to a station where he gets on a train. After finishing the train trip, he is picked up again by another car and is brought to his destination. It is also possible to change trains during this public transport trip. This behavior is very advantageous for the customer. While a partial train trip is cheaper than a pure car trip, the combination of cars and trains is faster than pure public transport since it does not require walking and transfer time.

\newpage

\paragraph{Assumption of Perfect Information} \parfill

As mentioned in \Cref{ch:introduction}, the introduction of autonomous vehicles probably involves a huge change in the customer behavior. The estimation of this is not part of the thesis, the focus lies rather in the potential of autonomous cars. Therefore, we use renting data from present time in order to model the customer behavior. The determined results for the optimal fleet size with autonomous vehicles can then be used to compare it to the current fleet size with conventional vehicles in order to estimate the possible enhancements. Further, we do not try to model an online behavior for the routing where the customer requests arise during during runtime. We rather assume to have perfect information of the customer behavior. This means we know all the travel requests in advance and create a vehicle schedule for the complete instance.

%----------------------------------------------------------------------------------------------------------------------------------------

\section{Problem Description and Notation}
\label{sec:problem_description}

We aim to state the problem for the routing of autonomous vehicles such that it suits to the situation as described before. In order to realize this, we introduce a formal notation. We are given a set of customers where we know the travel requests for each of them. Each of these travel requests can be realized by one of a set of alternative multimodal routes that is also given in advance. Each multimodal route consists of a sequence of trips. In this context, a trip is either a car trip or a public transport trip and has a fixed start and end position as well as a fixed start and end time and is completed without interruption and with the same means of transport for the whole duration. Fulfilling a route means that the customer takes all the trips of this route in a row, this means he starts at the start point of the first trip and is finished at the end point of the last trip. The transition between two subsequent trips is the changing from a car to a train or the other way round. Each customer has to be satisfied, this means it is possible that he fulfills one of his alternative routes. We call the constraints ensuring the customer satisfaction \enquote{cover constraints}.

Our goal is to create a schedule for the vehicles of the car sharing supplier. For this we assume that there is already a schedule for public transport fixed. The car trips are created in such a way that they suit to the public transport schedule which we describe in \Cref{ch:instance_creation}. Therefore we assume the given routes to be feasible. Since we are interested in a schedule for the cars and the car trips are chosen appropriately, the public transport trips are not part of the input. 

For fulfilling the trips, we have a set of vehicles. For each vehicle, we have a position where it starts and a time from when it is available. A vehicle can drive from its start point to a trip's start point, executing this trip, and then drive from the trip's end point to the next trip's start point. After fulfilling its last trip, the car stays at the end point of the last trip. The sequence, in which the vehicle executes the trips, is called the duty of the vehicle. 

A further restriction to the problem is that the vehicles have a maximal range. They can drive only a certain distance without refueling. We are given a set of refuel points where each car is able to refuel between serving trips. We call the constraints ensuring feasible fuel states for the vehicles \enquote{fuel constraints}.

\paragraph{Choice of the Multimodal Routes} \parfill

As mentioned before, each customer has a set of multimodal routes one of which has to be fulfilled. In real life could be assumed that the customer chooses his alternative on its own, for example according to time reasons, costs or his personal preferences. In contrast, we assume in this context that the choice of the route is made by the system. This means each customer takes exactly this route which is necessary for the overall schedule to be optimal. This behavior can be interpreted in two ways: Either this problem setting aims to find the best possible schedule under assumption of ideal customer decisions or the car sharing supplier presents only one alternative to the customer for his travel request.

As described later more formally, we introduce costs for the route choice into the model. With these route costs we can model the customer preferences. The customer preferences contain for example the total travel time, the number of changes or the cost for the customer. They work as penalty costs for inconvenient route choices. This means a route that is disadvantageous for the customer is penalized. Then either the for the customer less favorable route is chosen if it fits better into the schedule and is penalized. Or the more favorable route is chosen although it suits not so good in the schedule.

We can further add the costs for public transport to the route costs. Although we regard the problem solely from the supplier's point of view, we motivate this approach as follows: If we assume that each customer chooses a pure public transport route, this schedule would be optimal since there arise no costs for the supplier. Obviously, this is not optimal in reality since the supplier achieves no profit. If we try to maximize the profit, the optimal schedule would contain long car distances, although a customer would not pay for this in reality. Thus we include the customer costs in the model.

In summary, we try to model additional customer costs and customer inconveniences as penalty terms in order to receive a more realistic customer behavior.

\paragraph{Customers, Trips and Vehicles} \parfill

We first define the car trips, the multimodal routes, the customers and the connections between them. 

\begin{definition}[Trips, routes and customers]

\begin{enumerate}
	\item
We are given a set of car trips $\mathcal{T}$. Each trip $t\in\mathcal{T}$ has a start and end location $\pstart_t, \pend_t$ and a start and end time $\zstart_t, \zend_t$.

	\item
Further, we are given a set of multimodal routes $\mathcal{M}$. A route${m=\left(t_1,\dots,t_{k_m}\right)\in\mathcal{M}}$ is a finite sequence of trips with the following properties:
\begin{align*}
	\pend_{t_i} = \pstart_{t_{i+1}} && \zend_{t_i} \leq \zstart_{t_{i+1}} && \text{for all } i\in[k_m-1].
\end{align*}	
We define the route start and end locations and times for $m\in\mathcal{M}$ as
\begin{align*}
	\pstart_m := \pstart_{t_1} & &  \pend_m := \pend_{t_{k_m}} & & \zstart_m := \zstart_{t_1} & & \zend_m := \zend_{t_{k_m}}.
\end{align*}

	\item
We are given a set of customers $\mathcal{C}$. Each customer $c\in\mathcal{C}$ has a finite set of alternative multimodal routes.

	\item
Each trips belongs to exactly one route and each route belongs to exactly one customer. The mapping $M:\mathcal{T}\to\mathcal{M}$ indicates to which route a trip belongs and the mapping $C:\mathcal{M}\to\mathcal{C}$ shows to which customer a route belongs.	

	\item
For each route of the same customer $m\in C^{-1}(c)$, the start and end positions are the same, but the start and end times may differ. We define the customer start and end times for $c\in\mathcal{C}$
\begin{align*}
	\zstart_c := \min_{m\in C^{-1}(c)}\zstart_m & & \zend_c := \max_{m\in C^{-1}(c)}\zend_m .
\end{align*}
\end{enumerate}

\end{definition}

We use the following notation in order to describe the preimages of the mappings $M$ and $C$
\begin{align*}
	C^{-1}(c) & := C^{-1}\left(\{c\}\right) = \left\{m\in\mathcal{M}\mid C(m)=c\right\} && \text{for } c\in\mathcal{C} \\
	M^{-1}(m) & := M^{-1}\left(\{m\}\right) = \left\{t\in\mathcal{T}\mid M(t)=m\right\} && \text{for } m\in\mathcal{M} \\
	\left(M\circ C\right)^{-1}(c) & := M^{-1}\left(C^{-1}(c)\right) = \left\{t\in\mathcal{T}\mid C\left(M(t)\right)=c\right\} && \text{for } c\in\mathcal{C}
\end{align*}

for all routes of a customer, all trips of a route and all trips of a customer, respectively.

The vehicles that are needed for fulfilling the trips are introduced as follows:

\begin{definition}[Vehicles]

We are given a set of vehicles $\mathcal{V}$. For each vehicle $\vinV$ we are given a start position $p_v$ and a start time $z_v$.

\end{definition}

\paragraph{Fuel and Refueling} \parfill

We have to consider fuel restrictions. Fuel can be any form of energy with which the considered vehicle is powered. For each vehicle, the fuel level is in the interval $[0,1]$, where~$1$ means full capacity and~$0$ is empty. We call a drive without a customer, \ie a drive between two trips, a deadhead trip. A car may visit a refuel station only during a deadhead trip. For simplicity of the model, each car is allowed to refuel at most once between two trips. On a refuel station, there are no capacity constraints, \ie two or more vehicles may refuel at the same time at the same station. We define the refuel points and the fuel consumption as follows:

\begin{definition}[Fuel and refuel points]

\begin{enumerate}
	\item
We are given a set of refuel stations $\mathcal{R}$. Each refuel station $r\in\mathcal{R}$ has a location $p_r$.

	\item
We define $\fd_{s,t}$ for ${s\in\mathcal{V}\cupdot\mathcal{T}\cupdot\mathcal{R}}, {t\in\mathcal{T}\cupdot\mathcal{R}}$ as the amount the fuel level decreases along the deadhead trip between $s$ and $t$. We define $\ft_t$ for ${t\in\mathcal{T}}$ as the amount of fuel a vehicle needs for a trip and $f^0_v$ for $\vinV$ as the initial fuel state of a vehicle.
\end{enumerate}

\end{definition}

The amount of fuel that is charged at a refuel point between two trips can be determined from the time the vehicle stays at the refuel point between these trips. \fxnote{ft for r in chapter 3}

\paragraph{Ordering of the Trips} \parfill

We define the time a vehicle needs to get from position $p_1$ to $p_2$ as $t_{p_1,p_2}$. We define
\begin{equation*}
	t_{s,t} = 
	\begin{cases}
		t_{\pend_s,\pstart_t} & \text{if } s,t\in\mathcal{T} \\
		t_{p_s,\pstart_t} & \text{if } s\in\mathcal{V}\cupdot\mathcal{R}, t\in\mathcal{T} \\
		t_{\pend_s,p_t} & \text{if } s\in\mathcal{T}, t\in\mathcal{R} \\
		t_{p_s,p_t} & \text{if } s\in\mathcal{V},t\in\mathcal{R}
	\end{cases}
\end{equation*}
as the time a vehicle needs from one trip to another.

In order to decide whether a vehicle is able to fulfill two trips in a row, we define a partial ordering on the set of vehicles and trips. The set of public transport trips is left out in this definition.

\begin{definition}[Order of trips]
\label{def:partial_order}

The binary relation $\prec$ on $\mathcal{V}\cupdot\mathcal{T}$ is defined as follows:
\begin{align*}
	& s\prec t \quad :\Leftrightarrow \quad z_s + t_{s,t} \leq \zstart_t && \text{for all } s\in\mathcal{V}, t\in\mathcal{T} \\	
	& \omit\rlap{$\displaystyle{s\prec t \quad :\Leftrightarrow \quad \left(\zend_s + t_{s,t} \leq \zstart_t\right)\land\left((M\circ C)(s)\neq (M\circ C)(t)\lor M(s)=M(t)\right)}$} \\
	&&& \text{for all } s\in\mathcal{T}, t\in\mathcal{T} \\
	& s\not\prec t && \text{for all } s\in\mathcal{V}\cupdot\mathcal{T}, t\in\mathcal{V}
\end{align*}

The binary relation $\preceq$ on $\mathcal{V}\cupdot\mathcal{T}$ is defined as:
\begin{align*}
	s\preceq t \quad :\Leftrightarrow \quad s=t \lor s\prec t && \text{for all } s,t\in\mathcal{V}\cupdot\mathcal{T}
\end{align*}

\end{definition}

The expression $s\prec t$ means that one car is able to fulfill both trips, first $s$ and then $t$. A car must not cover two trips of the same customer, except when they belong to the same route. This results from the problem description, where for each customer exactly one route is fulfilled. 

\begin{remark}

Note that $\preceq$ is not a partial order on $\mathcal{V}\cupdot\mathcal{T}$ since the transitivity is missing. Let $t_1,t_2,t_3\in\mathcal{T}$ with 
\begin{align*}
	\zend_{t_1} + t_{t_1,t_2} \leq \zstart_{t_2} && \zend_{t_2} + t_{t_2,t_3} \leq \zstart_{t_3}
\end{align*}
\begin{align*}
	(M\circ C)\left(t_1\right) = (M\circ C)\left(t_3\right) && (M\circ C)\left(t_1\right) \neq (M\circ C)\left(t_2\right) && M\left(t_1\right)\neq M\left(t_3\right)
\end{align*}

Then we have
\begin{align*}
	t_1 \preceq t_2 \preceq t_3 && t_1 \not\preceq t_3
\end{align*}

\end{remark}

\paragraph{Problem Description and Cost} \parfill

Using the previously introduced notation, we can formally define the problem of routing autonomous vehicles with fuel constraints and multi-leg cover constraints:

\begin{definition}[Feasible schedule]
\label{def:feasible_schedule}

A feasible schedule is an assignment of vehicles~$\mathcal{V}$ to trips~$\mathcal{T}$ including refuel points~$\mathcal{R}$ with the following properties:

\begin{enumerate}
	\item
Each vehicle $\vinV$ has a feasible duty ${d_v := \left(t_1,\dots,t_{k_v}\right)}$ with ${t_i\in\mathcal{T}}$ and
\begin{align*}
	v\prec t_1 && t_i\prec t_{i+1} && \text{for all } i\in[k_v]
\end{align*}

Before each trip a refuel point can be visited. The fuel state of $v$ considering $f^0, \ft, \fd$ and the respective refueling is always in the range~$[0,1]$. The empty duty is feasible.

	\item
For each customer $c\in\mathcal{C}$ exactly one route ${m\in C^{-1}(c)}$ is fulfilled. A route ${m\in\mathcal{M}}$ is fulfilled if all its trips are fulfilled, \ie
\begin{align*}
	t\in\Cupdot_{v\in\mathcal{V}} d_v && \text{for all } t\in M^{-1}(m)
\end{align*}
\end{enumerate}

\end{definition}

After defining feasible schedules we state the costs that arise for it.

\begin{definition}[Costs of a schedule]
\label{def:costs}

Consider a feasible schedule according to \Cref{def:feasible_schedule}. We define the following types of costs:

\begin{enumerate}
	\item{Vehicle cost $\cv$: unit cost for each vehicle used}
	\item{Deadhead cost $\cd_{s,t}$ for $s\in\mathcal{V}\cupdot\mathcal{T}\cupdot\mathcal{R}, t\in\mathcal{T}\cupdot\mathcal{R}$:\\
	cost if a vehicle drives to a trip or a refuel station without a customer using it}
	\item{Trip cost $\ct_t$ for $t\in\mathcal{T}$: cost for fulfilling a trip}
	\item{Route cost $\croute_m$ for $m\in\mathcal{M}$: cost for fulfilling a route}
\end{enumerate}

\end{definition}

The vehicle costs are unit costs for each vehicle that is used to fulfill trips. If a vehicle does not serve any trips, \ie its duty is empty, then no vehicle costs occur for this vehicle. The deadhead costs arise for all the deadhead trips that some vehicle takes. The trip costs and route costs arise if this trip or route is fulfilled.

In summary, we formalize our problem as follows: Find a feasible schedule according to \Cref{def:feasible_schedule} such that the overall cost of the schedule as defined in \Cref{def:costs} is minimized.

\paragraph{Additional Assumptions} \parfill

In the following, we summarize all the assumptions we made on the input data.

All costs are non-negative.
\begin{align}
\label{eq:nonnegativ_cost}
	\cv\geq 0 && \cd_{s,t}\geq 0 && \ct_t\geq 0 && c^{\operatorname{r}}_m\geq 0 && \text{for all } s,t\in\mathcal{T}, m\in\mathcal{M}.
\end{align}

The fuel consumption is non-negative, except for refueling.
\begin{align}
	& \ft_t\geq 0 && \ft_r\leq 0 && \text{for all } t\in\mathcal{T}, r\in\mathcal{R} \\
	& \fd_{s_1,s_2} \geq 0 &&&& \text{for all } s_1\in\mathcal{V}\cupdot\mathcal{T}\cupdot\mathcal{R},s_2\in\mathcal{T}\cupdot\mathcal{R}
\end{align}

There are no zero-time rentals.
\begin{align}
\label{eq:notime_trip}
	\zstart_t < \zend_t && \text{for all } t\in\mathcal{T}
\end{align}

We assume the Triangle Inequalities for time, fuel consumption and cost. For $s\in\mathcal{V}\cupdot\mathcal{T}\cupdot\mathcal{R}$ and $r,t\in\mathcal{T}\cupdot\mathcal{R}$ holds:
\begin{align}
\label{eq:triangle_inequality}
	t_{s,t} \leq t_{s,r} + t_{r,t} && \cd_{s,t} \leq \cd_{s,r} + \cd_{r,t} && \fd_{s,t} \leq \fd_{s,r} + \fd_{r,t}
\end{align}

From $\eqref{eq:nonnegativ_cost}$ - $\eqref{eq:triangle_inequality}$ we get for all $s,r,t\in\mathcal{T}$:
\begin{align}
	t_{s,t} & \leq t_{s,r} + \left(\zend_r - \zstart_r\right) + t_{r,t} \label{eq:triangle_inequality_time} \\
	\fd_{s,t} & \leq \fd_{s,r} + \ft_{r} + \fd_{r,t} \label{eq:triangle_inequality_fuel} \\
	\cd_{s,t} & \leq \cd_{s,r} + \ct_{r} + \cd_{r,t} \label{eq:triangle_inequality_cost}
\end{align}

%----------------------------------------------------------------------------------------------------------------------------------------

\section{Classification}
\label{sec:classification}

We aim to classify our problem in relation to other known problems in the literature and state the difficulty of these problems.

\paragraph{Vehicle Scheduling Problems} \parfill

According to the structure of the problem stated in \Cref{sec:problem_description}, we regard the field of vehicle scheduling problems (VSP). \cite{Bunte_Kliewer} define the VSP as follows: \enquote{Given a set of timetabled trips with fixed travel (departure and arrival) times and start and end locations as well as traveling times between all pairs of end stations, the objective is to find an assignment of trips to vehicles such that each trip is covered exactly once, each vehicle performs a feasible sequence of trips and the overall costs are minimized.} The complexity of some variants of the VSP is regarded by \cite{Lenstra_Kan}.

A similar problem formulation is the dial-a-ride problem (DARP). \cite{Cordeau_Laporte} discuss the differences of the DARP to other vehicle routing problems and write: \enquote{What makes the DARP different from most such routing problems is the human perspective. When transporting passengers, reducing user inconvenience must be balanced against minimizing operating costs.} The basic formulations of VSP and DARP are the same, therefore we use the formulation of VSP as it is more common.

\paragraph{Depot Variants} \parfill 

In \cite{Bunte_Kliewer}, there are two main variants for the VSP with respect to where vehicles start and return to. In the single depot case (SD-VSP), there is one depot from where all vehicles start. After usage, all vehicles return to this depot. The multiple depot case (MD-VSP) means that there is more than one depot and from each depot a certain number of vehicles starts. After usage, each vehicle returns to the depot from where it has started. 

In order to make our problem more realistic, we model more than one depot. There are more than one vehicle where each vehicle starts at its specific start position. The vehicles do not have a certain point where they have to return to after usage, \ie they can stay wherever the last trip of their duty ends. \cite{Daduna_Paixao} claim that \enquote{if the vehicle[s] are allowed to return to a depot different from its origin depot, [...] the problem can be solved as a single depot instance.} We see that our problem is in the single depot case SD-VSP concerning the depot variant.

In \cite{Dantzig_Fulkerson}, it is proven that SD-VSP can be solved in polynomial time, \ie SD-VSP is in $\mathcal{P}$. In contrast, the multiple depot case MD-VSP is $\mathcal{NP}$-hard as shown in \cite{Bertossi_Carraresi}.

\newpage

\paragraph{Fuel Constraints} \parfill

We further consider fuel constraints in our problem. The literature (cf. \cite{Bunte_Kliewer}, \cite{Raff}) names general resources like time, mileage or fuel, summarized in the general term \enquote{route constraints}. The respective problems with route constraints are called SD-VSP-RC and MD-VSP-RC. \cite{Freling_Paixao} describe  the VSP with time constraints, \cite{Raff} present the VSP with path constraints, which is a more general formulation. In these models, a vehicle returns to the depot after the respective resource is exhausted, while the vehicle has the possibility to refuel at certain locations in our model. The problem with not refilling the resource is a special case of the problem with the possibility to refill the resource. We see in \Cref{sec:complexity} that SD-VSP-RC is already $\mathcal{NP}$-hard.

There are two approaches of the VSP including refueling stations in the literature, namely the Alternative Fuel VSP (AF-VSP) introduced by \cite{Adler} and the Electric VSP (E-VSP) introduced by \cite{Wen}. The E-VSP is defined as a \enquote{Multi-Depot VSP with distance constraints and charging possibilities. [...] Each vehicle can be recharged fully or partially at any given recharging station.} \cite[p.~73]{Wen}. If we identify the distance constraints with the fuel constraints, this formulation comes close to our problem setting. The differences to the E-VSP are that we regard the single-depot case and add cover constraints. The AF-VSP further only allows full recharging and assumes a constant charging time, where partial recharging is possible in our problem and the charging time depends on the remaining fuel state.

\paragraph{Cover Constraints} \parfill

In the basic SD-VSP, all existing trips have to be fulfilled. In the underlying master theses, only a selection of the trips has to be fulfilled which provides a more general setting. Namely, there are \enquote{customers with sets of alternative trips out of which exactly one trip shall be fulfilled, respectively.} (\cite[p.~10]{Kaiser}, \cite[p.~10]{Knoll}) In our problem, even more general cover constraints are required. There are customers with sets of alternative routes, consisting of trips; for each customer, exactly one route has to be fulfilled, \ie each of its trips is fulfilled. We call these constraints \enquote{multi-leg} cover constraints and write the problem VSP-MC. This is a generalization to the previous cover constraints, as can be seen easily by rewriting them: There are customers with sets of alternative routes, where each route consists of exactly one trip. According to this reformulation, we call the primary constraints \enquote{single-leg} cover constraints and write the problem VSP-SC. We see in \Cref{sec:complexity} that VSP-SC is already $\mathcal{NP}$-hard.

\newpage

\paragraph{Conclusion} \parfill

In summary, we have a problem with two types of constraints which individually make the problem $\mathcal{NP}$-hard, namely the multi-leg cover constraint and the fuel constraint. The only difference to \cite{Kaiser} and \cite{Knoll} is that the single-leg cover constraint is replaced by the multi-leg cover constraint. Their solution methods are extended to our requirements in this thesis. To the knowledge of the author, these cover constraints have not been treated in the literature. Further, there is no appearance of cover constraints in general in combination with vehicle scheduling in the literature, besides the underlying theses. 

%----------------------------------------------------------------------------------------------------------------------------------------

\section{Complexity}
\label{sec:complexity}

We regard the complexity of our problem. As we have seen in \Cref{sec:classification}, the problem can be modeled as a single depot vehicle scheduling problem SD-VSP with resource constraints, the possibility of refueling and multi-leg cover constraints. The SD-VSP itself can be solved in polynomial time, which is proven by \cite{Dantzig_Fulkerson}. It can be formulated as a minimum-cost flow problem. If we extend the basic formulation with one of the additional constraints, it becomes $\mathcal{NP}$-hard.

\begin{theorem}[VSP with resource constraints]
\label{th:complexity_VSPRC}

The vehicle scheduling problem with resource constraints and the special objective of minimizing the number of used vehicles is $\mathcal{NP}$-hard.

\end{theorem}

This theorem is proven in \cite[p.~11]{Kaiser} and\cite[p.~11]{Knoll} by a polynomial reduction of the bin packing problem. This theorem holds for the case of resource constraints without the possibility of refilling the resources. Since this is a special case of resource constraints with refueling, this problem is also $\mathcal{NP}$-hard. 

\begin{theorem}[VSP with cover constraints]
\label{th:complexity_VSPSC}

The vehicle scheduling problem with cover constraints and the special objective of minimizing the number of used vehicles is $\mathcal{NP}$-hard.

\end{theorem}

This theorem is proven in \cite[p.~12]{Kaiser} and \cite[p.~12]{Knoll} by a polynomial reduction of the set cover problem. This theorem treats the case of single-leg cover constraints.

\begin{theorem}[VSP with multi-leg cover constraints]
\label{th:complexity_VSPMC}

The vehicle scheduling problem with multi-leg cover constraints and the special objective of minimizing the number of used vehicles is $\mathcal{NP}$-hard.

\end{theorem}

\begin{proof}

We prove this statement by a polynomial reduction of the VSP-SC. Consider a VSP-SC with vehicles $\hat{\mathcal{V}}$, customers $\hat{\mathcal{C}}$, trips $\hat{\mathcal{T}}$ and the function $\hat{C}:\hat{\mathcal{T}}\to\hat{\mathcal{C}}$ that maps trips to their respective customers. The problem is defined in detail in \cite{Kaiser} and \cite{Knoll}. We create the corresponding VSP-MC as follows: $\mathcal{V}:=\hat{\mathcal{V}}$, $\mathcal{C}:=\hat{\mathcal{C}}$, $\mathcal{T}:=\hat{\mathcal{T}}$ stay the same. We define the multimodal routes as one-element sequences for each trip
\begin{align*}
	\mathcal{M}:=\left\{(t)\mid t\in\hat{\mathcal{T}}\right\}
\end{align*}
and the mappings
\begin{align*}
	M:\mathcal{T}\to\mathcal{M} \text{, } M(t) = (t), && C:\mathcal{M}\to\mathcal{C} \text{, } C(m) = \hat{C}(t) && \text{for } t\in m \text{ unique.}
\end{align*}

The solution of VSP-SC can easily be mapped to a solution of the corresponding VSP-MC and vice versa. If a trip is chosen in VSP-SC, the respective route is chosen in VSP-MC. Then, every trip of this route is fulfilled and the customer is satisfied. The other way round, if a route is chosen in VSP-MC, the trip contained in this route is chosen in VSP-SC and the customer is satisfied. The feasibility of the vehicle duties is not affected by this procedure.

This is a polynomial reduction, VSP-SC is $\mathcal{NP}$-hard by \Cref{th:complexity_VSPSC} and hence VSP-MC is $\mathcal{NP}$-hard.

\end{proof}

With \Cref{th:complexity_VSPRC} and \Cref{th:complexity_VSPMC} we can see that our problem gets $\mathcal{NP}$-hard only with cover constraints or with resource constraints and considering the number of vehicles.

The problem becomes even harder, as we do not only consider the number of vehicles but include the operational cost and penalty terms for customer preferences. Further, we want to have the possibility to refuel during the process, \ie we have negative resource cost. Finally, we aim to apply both of these constraints simultaneously. 

We will model our problem as a mixed-integer linear program in a size polynomial in the input size. Therefore, it is possible to verify a solution in polynomial time. This means our problem is $\mathcal{NP}$-complete.

%----------------------------------------------------------------------------------------------------------------------------------------

\section{Approaches in the Literature}

In the previous sections we have seen that the problem we are dealing with is very hard. We cannot expect to find an algorithm that solves the problem optimally running in polynomial time. Therefore we focus on the field of heuristical solution methods. Using them it is possible to find a good solution for realistic instances in reasonable time. The field of the vehicle scheduling problems is highly researched and there are many solution approaches available in the literature. \cite{Bunte_Kliewer} provide an extensive overview of the solution approaches for both problems SD-VSP and MD-VSP. We classify our problem as a SD-VSP with resource constraints and multi-leg cover constrains. This is only a small extension to the problem treated in the underlying theses where single-leg cover constraints are contained. As \cite{Kaiser} and \cite{Knoll} already discuss the concerning approaches in the literature, we refer to them for more details. In the following, we present outlines of the solution methods for the AF-VSP and the E-VSP.

\paragraph{Heuristical Approaches} \parfill

In order to tackle the AF-VSP, \cite{Adler} present a Greedy heuristic. The trips are iteratively added to the vehicle duty that gives the lowest cost-increasing provided that the duty stays feasible. During the feasibility check, also the respective refuel points are inserted to the duties. A new vehicle duty is created if no feasible appending is possible. This heuristic is fast but does not provide good results. Therefore, it is only appropriate for computing an initial solution.

\cite{Wen} provide an Adaptive Large Neighborhood Search heuristic for the E-VSP. We present a rough outline of this algorithm. First, an initial solution is created by a greedy heuristic. From this solution, a certain number of trips is deleted (destroy method). After this, several new feasible duties are created by inserting respective trips with lowest cost-increasing at certain positions (repair method). Finally, the new duties are selected with minimal total cost, such that the duties cover each trip and the depot constraints are fulfilled. Several destroy and repair methods are presented there as well as a deterministic and a randomized version of the heuristic.

\paragraph{Optimal Approach} \parfill

For an optimal solution of the AF-VSP, \cite{Adler} develop a Branch-and-Price procedure. A path flow formulation is considered which yields a master problem defining a set partitioning problems. The relaxed version of this master problem is solved via column generation. The subproblems for finding new feasible duties is a Weight Constrained Shortest Path Problem with Replenishment. As branching decisions, they use the fact if a certain trip is directly fulfilled after another or not. They use a Greedy heuristic for an initial solution and compute an optimal solution with this Branch-and-Price method.

\paragraph{Modeling the Refuel Points} \parfill

There is no straightforward procedure to model the refuel points. It is not possible to use them individually in a task graph since the single vehicle duties cannot be distinguished if several duties visit the same refuel point. We tackle this issue by creating copies of the refuel points.

\cite{Adler} create a refuel point node after each depot and after each trip for each refuel point. These nodes represent the respective visit of the refuel point at the beginning of a duty and directly after a trip. It is not necessary to specify the charging time at the refuel point since the recharging time is assumed constant there. If we identify the depots with the vehicles in our problem, we have ${\left(\vert\mathcal{V}\vert + \vert\mathcal{T}\vert\right)\vert\mathcal{R}\vert}$ refuel point copies in the task graph. As we do not assume a constant charging time, we cannot directly apply this approach to our problem.

\cite{Wen} create two copies for each refuel point and trip. One refuel point node is set between the depots and the trip, the other one is set after the trip. With this, they guarantee that a vehicle is able to refuel before the trip if the duty starts with this trip and after the trip. If we identify the depots with the vehicles in our problem, there are ${2\vert\mathcal{T}\vert\vert\mathcal{R}\vert}$ refuel point copies in the task graph. Since partial refueling is allowed there, it is necessary to determine the respective refueling times. This is realized by an additional time variable in the formulation. In order to maintain this variable, a complete graph is necessary. Thus, a preprocessing on the graph, which eliminates edges between time-infeasible trips and is common usage for the VSP, is not possible there. 

In the underlying theses, \cite{Kaiser} and \cite{Knoll} create a copy between each feasible pair of trips for each refuel point. With this method it is possible to determine the respective charging time individually in advance and modeling partial refueling is easy. In the task graph, time-infeasible arcs are removed in advance and are not treated by additional constraints. In the worst case, there are ${\left(\frac 1 2 \vert\mathcal{T}\vert^2 + \vert\mathcal{V}\vert\vert\mathcal{T}\vert\right)\vert\mathcal{R}\vert}$ refuel point copies in the task graph. In order to reduce the size of the task graph, a preprocessing on the refuel point nodes based on Pareto-optimality is provided. Using this for realistic instances, most of the copies can be eliminated without cutting the optimal solution.

\paragraph{Applications} \parfill

Besides car sharing with autonomous vehicles, there are further applications of this kind of problem in the literature, namely the routing of buses or the transport of cargo (cf. \cite{Wen}, \cite{Adler}). The VSP in general is suitable if the time in which the respective customers are served is important. The VSP with fuel constraints is particularly applied for vehicles with limited fuel capacity and a scarce number of available refuel points, such as electric vehicles. The cover constraints can be used if the serving of a subset of the requests suffices, for example if customers have several alternatives. To the knowledge of the author, this issue has not been treated in the literature.