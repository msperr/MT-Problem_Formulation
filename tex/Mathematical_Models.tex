\chapter{Mathematical Models}
\label{ch:mathematical_models}

We introduce the mathematical model, with which we want to solve the previously described problem. We define the underlying task graph and develop an arc flow formulation on this graph. The main idea is to model a flow on the vehicles and the trips. This flow has to fulfill additional requirements such that the cover constraints and the fuel constraints are fulfilled. As mentioned before, the public transport trips are not part of the input. We therefore consider only the car trips in the mathematical model.

\section{Task Graph}

We introduce the task graph, on which the model is based. It is a directed graph based on the relation $\prec$ according to \Cref{def:partial_order}, \ie there is an edge $(s,t)$ in the graph if $s\prec t$ holds. The graph is basically the same as used in \cite{Kaiser} and \cite{Knoll} with the only difference that the customer and route considerations are adapted here.

\begin{definition}[Task Graph]
\label{def:taskgraph}

Let $\ds,\de$ be special vertices describing the source and sink of the vehicle flow. We define the task graph as $G=(V,A)$, where
\begin{align*}
	V:=\left\{\ds,\de\right\}\cupdot\mathcal{V}\cupdot\mathcal{T}
\end{align*}

is the vertex set consisting of the source, the sink, the vehicle set $\mathcal{V}$ and the trip set $\mathcal{T}$. The arc set is
\begin{align*}
	A & :=\left(\left\{\ds\right\}\times\mathcal{V}\right)\cupdot\left\{(s,t)\in\left(\mathcal{V}\cupdot\mathcal{T}\right)^2|s\prec t\right\}\cupdot\left(\left(\mathcal{V}\cupdot\mathcal{T}\right)\times\left\{\de\right\}\right).
\end{align*}

\end{definition}
\nomenclature[ds]{$\ds$}{source of the flow}
\nomenclature[de]{$\de$}{sink of the flow}
\nomenclature[ggraph]{$G=\left(V,A\right)$}{task graph}
\nomenclature[vgraph]{$V$}{vertex set of the task graph}
\nomenclature[agraph]{$A$}{arc set of the task graph}

A vertex $s\in\mathcal{V}$ represents the initial state of a vehicle $s$ where it becomes available for the first time. Each $\ds$-$\de$-path in $G$ is the duty of one vehicle, i.e. this vehicle fulfills the trips in the order given by the path. Hence, two trips are connected only if it is possible that one car fulfills both trips, i.e. the relation $\prec$ holds.

\begin{lemma}[cf. \cite{Kaiser}, \cite{Knoll}]
\label{lem:taskgraph_cyclefree}

$G$ is a directed acyclic graph.

\end{lemma}

\begin{proof}

Assume there is a cycle in $G$. The source $\ds$ and sink $\de$ have only ingoing, respectively outgoing arcs and are therefore not part of the cycle. For $\vinV$, all ingoing arcs come from $\ds$, hence $\vinV$ are neither part of the cycle. This means, a cycle consists only of trips.

Consider an arbitrary cycle of trips $t_1,\dots,t_k\in\mathcal{T}$, $k\geq 2$. These trips form a cycle, \ie ${t_1\prec\dots\prec t_k}$ and ${t_k\prec t_1}$. With \Cref{def:partial_order} and the assumptions $\eqref{eq:notime_trip}$ and $\eqref{eq:triangle_inequality}$ holds:
\begin{gather*}
	\zstart_i < \zend_i \leq \zend_i + t_{t_i,t_{i+1}} \leq \zstart_{i+1} < \zend_{i+1} \qquad\qquad \text{for all } i\in[k-1] \\
	\Rightarrow \qquad \zstart_1 < \zend_k \qquad \Rightarrow \qquad \zend_k + t_{t_1,t_k} > \zstart_1 \qquad \Rightarrow \qquad t_k\not\prec t_1
\end{gather*}

This is a contradiction. Therefore no cycle exists.
%
\end{proof}

In order to consider refueling and refuel stations, we introduce an extended task graph.

\begin{definition}[Extended Task Graph]
\label{def:extended_taskgraph}

For every $s,t\in\mathcal{V}\cupdot\mathcal{T}$ with $s\prec t$ we create a copy of $\left\{r\in\mathcal{R}|\zend_s+t_{s,r}+t_{r,t}\leq\zstart_t\right\}$ denoted by $\Rst$. This means, various copied sets are pairwise disjoint. The expression $r\in\Rst$ means that a vehicle is able to finish trip $s$, then drive to refuel station $r$ and then start trip $t$ in time.
\nomenclature[refuelpoint]{$\Rst$}{set of refuel points on arc $(s,t)$}

We define the extended task graph $\widehat{G}=\left(\widehat{V},\widehat{A}\right)$ with vertex set
\begin{align*}
	\widehat{V}:=V\cupdot\Cupdot_{\substack{s,t\in\mathcal{V}\cupdot\mathcal{T} \\ s\prec t}} \Rst
\end{align*}

and arc set
\begin{align*}
	\widehat{A}:=A\cupdot\left\{(s,r)|s,t\in\mathcal{V}\cupdot\mathcal{T},s\prec t,r\in\Rst\right\}\cupdot\left\{(r,t)|s,t\in\mathcal{V}\cupdot\mathcal{T},s\prec t,r\in\Rst\right\}.
\end{align*}

It is possible that there is a copy of each refuel station for each feasible pair of trips. This leads to an enormous size of the task graph and thus a bad solution behavior is expected. \cite{Kaiser} and \cite{Knoll} describe a method to reduce the size of $\Rst$ without cutting the optimal solution. This method only considers Pareto-optimal refuel stations \wrt a suitable function. From now on, we will use $\widehat{G}=\left(\widehat{V},\widehat{A}\right)$ with restricted $\Rst$.

\end{definition}
\nomenclature[ggraph]{$\widehat{G}=\left(\widehat{V},\widehat{A}\right)$}{extended task graph}
\nomenclature[vgraph]{$\widehat{V}$}{vertex set of the extended task graph}
\nomenclature[agraph]{$\widehat{A}$}{arc set of the extended task graph}

\begin{lemma}[cf. \cite{Kaiser}, \cite{Knoll}]

$\widehat{G}$ is a directed acyclic graph.

\end{lemma}

\begin{proof}

Assume there is a cycle in $\widehat{G}$. In comparison to $G$, only arcs $(s,r)$ and $(r,t)$ for ${r\in\Rst, s\prec t}$ were added. Assume there is a cycle containing $r\in\Rst$. $r$ has only one ingoing arc $(s,r)$ and one outgoing arc $(r,t)$ and only if the arc $(s,t)$ exists. There is no cycle on the vertices $\left\{s,r,t\right\}$. Every other cycle containing $r$ is also a cycle using the arc $(s,t)$. This is a contradiction to the fact that $G$ is cycle-free as proven in \Cref{lem:taskgraph_cyclefree}.
%
\end{proof}

In the extended task graph $\widehat{G}$, a $\ds$-$\de$-path further represents the duty of a vehicle. The additional arcs $(s,r),(r,t)$ for $r\in\Rst$ describe a possible detour between the trips $s$ and $t$ in order to refuel at refuel station $r$.

We introduce the following frequently used notation:
\begin{align*}
	\Nin(t) := \left\{s\in V\mid (s,t)\in A\right\} && \Nout(s) := \left\{t\in V\mid (s,t)\in A\right\}
\end{align*}
\nomenclature[nin]{$\Nin(t)$}{set of in-neighbors}
\nomenclature[nout]{$\Nout(t)$}{set of out-neighbors}

$\Nin(t)$ is the set of in-neighbors of $t\in V$, $\Nout(s)$ is the set of out-neighbors of $s\in V$.

%---------------------------------------------------------------------------------------------------------------------------------------

\section{Arc Flow Formulation}
\label{sec:arcflow_formulation}

In the following, we model the problem via a flow of the vehicles on the task graph. This model is not based on the extended task graph. Instead of considering the refuel points explicitly in the flow, we rather use them implicitly with additional constraints. The trips and multimodal routes are given in advance. The fact whether two trips can be fulfilled subsequently in one duty, is already given by the underlying task graph. We additionally have to model the cover constraints and the fuel constraints. Since the duties of various vehicles are disjoint \wrt $\mathcal{T}$, we are able to use one common set of variables for the flow of the vehicles. From the flow, we can easily extract the individual $\ds$-$\de$-paths in order to identify the duties of the respective vehicles.

\paragraph{Basic Model} \parfill

We model the arc flow as a mixed-integer linear program. The formulation is basically built on the $\eqref{eq:MILP}$ formulation as described in (\cite[p.~34]{Kaiser}, \cite[p.~34]{Knoll}). We use the following decision variables:
\begin{itemize}
	\item{$x_{s,t}\in\{0,1\}$ for $(s,t)\in A$:} \\
	indicates, whether trip $t\in\mathcal{T}$ is fulfilled directly after ${s\in \mathcal{V}\cupdot\mathcal{T}}$
	\item{$z_{s,r,t}\in\{0,1\}$ for $t\in\mathcal{T},s\in\Nin(t),r\in\Rst$:} \\
	indicates, whether refuel station $r\in\mathcal{R}$ is visited between $s$ and $t$
	\item{$e_s\in[0,1]$ for $s\in V\backslash\left\{\ds,\de\right\}$:} \\
	states the fuel of the respective vehicle after fulfilling trip $s\in\mathcal{T}$
\end{itemize}

If $s\in\mathcal{V}$, then $x_{s,t}$ determines whether trip $t$ is the first trip fulfilled by $s$ and $e_s$ is the initial fuel state $f^0_s$ of vehicle $s$.

Additionally to $\eqref{eq:MILP}$, we introduce decision variables in order to ensure the cover constraints:

\begin{itemize}
	\item{$u_m\in\{0,1\}$ for $m\in\mathcal{M}$: indicates whether multimodal route $m$ is fulfilled}
\end{itemize}

The basic constraints are developed by \cite{Kaiser} and \cite{Knoll} and not explained in detail here. The basic constraints model a feasible flow of the vehicles, the refuel points and the fuel constraints. The cover constraints are developed now.

\paragraph{Cover Constraints} \parfill

In the $\eqref{eq:MILP}$ (cf.~\cite[p.~34]{Kaiser}, \cite[p.~34]{Knoll}), each customer has a set of alternative trips and from this set, exactly one trip has to be fulfilled. The mapping of trips to customers is given by ${\widehat{C}:\mathcal{T}\to\mathcal{C}}$ there. The cover constraints are modeled as follows:

\begin{align}
	\sum_{t\in \widehat{C}^{-1}(c)} \sum_{s\in\Nin(t)} x_{s,t} = 1 &&& \text{for all } c\in\mathcal{C} \label{eq:MILP:customer}
\end{align}

In contrast to the $\eqref{eq:MILP}$, each customer has a set of alternative routes consisting of trips here and from this set, exactly one route has to be fulfilled. Therefore, we replace $\eqref{eq:MILP:customer}$ by the following formulation:

\begin{align}
	\sum_{m\in C^{-1}(c)} u_m = 1 &&& \text{for all } c\in\mathcal{C} \label{eq:MMILP:customer} \\
	\sum_{s\in\Nin(t)} x_{s,t} = u_m &&& \text{for all } m\in\mathcal{M}, t\in m \label{eq:MMILP:route}
\end{align}

Constraint $\eqref{eq:MMILP:customer}$ ensures the completion of exactly one route of each customer. Constraint $\eqref{eq:MMILP:route}$ says, if a route is fulfilled then every trip of this route must be fulfilled.

\paragraph{Objective Function} \parfill

The objective function in the $\eqref{eq:MILP}$ is given by

\begin{align*}
	\sum_{s\in\mathcal{V}} \sum_{t\in\Nout(s)\backslash\{d^{\operatorname{e}}\}} x_{s,t}\cv	+ \sum_{t\in\mathcal{T}} \sum_{s\in\Nin(t)} \left[x_{s,t}\left(\cd_{s,t}+\ct_{t}\right) + \sum_{r\in\Rst} z_{s,r,t}\left(\cd_{s,r}+\cd_{r,t}-\cd_{s,t}\right)\right]
\end{align*}

considering the vehicle costs $\cv$, the trip costs $\ct_t$ for $t\in\mathcal{T}$ and the deadhead costs $\cd$. The first part ensures that for each vehicle covering some trip the vehicle cost is charged. The second part takes care of the costs for each trip and deadhead trip under consideration of the detours for refueling. What is missing, are the route-dependent costs $\croute$. Thus, we add the term

\begin{align*}
	\sum_{m\in\mathcal{M}} u_m \croute_m
\end{align*}

to the objective function. Putting all together, we get the following formulation, called $\eqref{eq:MMILP}$:

\begin{align}
	\min \quad & \sum_{s\in\mathcal{V}} \sum_{t\in\Nout(s)\backslash\{d^{\operatorname{e}}\}} x_{s,t}\cv + \sum_{m\in\mathcal{M}} u_m \croute_m \nonumber \\
	& \omit\rlap{$\displaystyle{ + \sum_{t\in\mathcal{T}} \sum_{s\in\Nin(t)} \left[x_{s,t}\left(\cd_{s,t}+\ct_{t}\right) + \sum_{r\in\Rst} z_{s,r,t}\left(\cd_{s,r}+\cd_{r,t}-\cd_{s,t}\right)\right]}$} \tag{MMILP} \label{eq:MMILP} \\
	\text{s.t.} \quad & \sum_{s\in\Nin(t)} x_{s,t} = \sum_{s\in\Nout(t)} x_{t,s} & & \text{for all } t\in V\backslash\left\{d^{\operatorname{s}},d^{\operatorname{e}}\right\} \label{eq:MMILP:flow} \\
	& \sum_{s\in\Nin(t)} x_{s,t} = 1 & & \text{for all } t\in\mathcal{V} \label{eq:MMILP:vehicles} \\
	& \sum_{m\in C^{-1}(c)} u_m = 1 && \text{for all } c\in\mathcal{C} \tag{\ref{eq:MMILP:customer}} \\
	& \sum_{s\in\Nin(t)} x_{s,t} = u_m && \text{for all } m\in\mathcal{M}, t\in m \tag{\ref{eq:MMILP:route}} \\
	& \sum_{r\in\Rst} z_{s,r,t} \leq x_{s,t} & & \text{for all } t\in\mathcal{T}, s\in\Nin(t) \label{eq:MMILP:refuel} \\
	& e_s \leq f_s^0 & & \text{for all } s\in\mathcal{V} \label{eq:MMILP:initial_fuel} \\
	& 0 \leq e_s - \sum_{r\in\Rst} z_{s,r,t}\fd_{s,r} & & \text{for all } t\in\mathcal{T}, s\in\Nin(t) \label{eq:MMILP:min_fuel} \\
	& e_t \leq 1 - \ft_t - \sum_{r\in\Rst} z_{s,r,t}\fd_{r,t} & & \text{for all } t\in\mathcal{T}, s\in\Nin(t) \label{eq:MMILP:max_fuel} \\
	& \omit\rlap{$\displaystyle{e_t \leq e_s - x_{s,t}\left(f_{s,t}^{\operatorname{d}}+f_t^{\operatorname{t}}\right) - \sum_{r\in\Rst} z_{s,r,t}\left(\fd_{s,r}+\ft_r+\fd_{r,t}-\fd_{s,t}\right) + \left(1-x_{s,t}\right)}$} \nonumber \\
	& & & \text{for all } t\in\mathcal{T}, s\in\Nin(t) \label{eq:MMILP:fuel_consumption} \\
	& x_{s,t}\in\{0,1\} & & \text{for all } (s,t)\in A \label{eq:MMILP:xst} \\
	& z_{s,r,t}\in\{0,1\} & & \text{for all } t\in\mathcal{T},s\in\Nin(t),r\in\Rst \label{eq:MMILP:zsrt} \\
	& e_s\in[0,1] & & \text{for all } s\in V\backslash\left\{d^{\operatorname{s}},d^{\operatorname{e}}\right\} \label{eq:MMILP:es} \\
	& u_m\in\{0,1\} && \text{for all } m\in\mathcal{M} \label{eq:MMILP:um}
\end{align}
\nomenclature[mmilp]{$\left(\operatorname{MMILP}\right)$}{initial arc flow formulation}

\newpage

\paragraph{Model Equivalence} \parfill

The problem formulation as developed in \Cref{ch:mathematical_models} is equivalent to the problem setting as stated in \Cref{sec:problem_description}. This means that each feasible schedule according to \Cref{def:feasible_schedule} corresponds to a solution of the mixed-integer linear program $\eqref{eq:MMILP}$ and vice versa. This fact is shown in the following theorem.

\begin{theorem}[Model Equivalence]
\label{thm:equivalence_setting_mmilp}

Let ${S=\left(x,z,e,u\right)}$ be a feasible solution of the $\eqref{eq:MMILP}$. We can transform $S$ into a schedule $S'$ that is feasible according to \Cref{def:feasible_schedule}. Let $S'$ be a feasible schedule according to \Cref{def:feasible_schedule}. Then $S'$ can be transformed into a feasible solution of the $\eqref{eq:MMILP}$. The objective value $\operatorname{val}(S)$ and the schedule cost $\operatorname{cost}\left(S'\right)$ according to \Cref{def:schedule_cost} coincide.

\end{theorem}

\begin{proof}

Let ${S=(x,z,u,e)}$ be a feasible solution of the $\eqref{eq:MMILP}$. We construct the schedule $S'$ as follows: For each ${v\in\mathcal{V}}$, create a duty $d(v)$. Set ${t_0:=v}$. Then iteratively append $\left(r_i,t_i\right)$ to the duty for the unique ${t_i\in\mathcal{T}}$ with ${x_{t_{i-1},t_i}=1}$. Set ${r_i:=r}$ for the unique ${r\in\mathcal{R}_{t_{i-1},t_i}}$ with ${z_{t_{i-1},r,t_i}=1}$. Set ${r_i:=\emptyset}$ if ${z_{t_{i-1},r,t_i}=0}$ for all ${r\in\mathcal{R}_{t_{i-1},t_i}}$. The duty ends when ${x_{t_i,\de}=1}$. If ${x_{v,\de}=1}$ then we have ${d(v)=\emptyset}$. For each ${c\in\mathcal{C}}$ we set ${m(c):=m}$ for the unique ${m\in C^{-1}(c)}$ with ${u_m=1}$. Due to the flow conservation $\eqref{eq:MMILP:flow}$ and the binary constraints $\eqref{eq:MMILP:xst}$ and $\eqref{eq:MMILP:zsrt}$, there is always a unique trip ${t_i\in\mathcal{T}}$ and at most one refuel point ${r_i\in\mathcal{R}_{t_{i-1},t_i}}$. The cover constraints $\eqref{eq:MMILP:customer}$ and the binary constraints $\eqref{eq:MMILP:um}$ ensure that there is always a unique route ${m\in C^{-1}(c)}$.

Due to the construction of the graph $G$ and the sets $\Rst$ for ${s\in\mathcal{V}\cupdot\mathcal{T}}$ with $s\prec t$, the time feasibility for $S'$ is fulfilled. During a trip or deadhead trip, the fuel level of a vehicle decreases monotonously. While visiting a refuel points, the fuel level increases. Therefore we observe that a vehicle has its minimal fuel state directly before a refuel point or after a trip and its maximal fuel state directly after a refuel point or at the beginning of the duty. Constraints $\eqref{eq:MMILP:min_fuel}$ and $\eqref{eq:MMILP:es}$ ensure the fuel feasibility for the minimal fuel states and constraints $\eqref{eq:MMILP:max_fuel}$ and $\eqref{eq:MMILP:initial_fuel}$ for the maximal fuel states. Constraint $\eqref{eq:MMILP:fuel_consumption}$ describes the fuel consumption between two trips. The cover constraints are directly fulfilled by $\eqref{eq:MMILP:customer}$ and $\eqref{eq:MMILP:route}$. Therefore $S'$ is a feasible schedule according to \Cref{def:feasible_schedule}.

Let $S'$ be a feasible schedule with $v\mapsto d(v)$ and $c\mapsto m(c)$. We construct a solution $(x,z,u,e)$ as follows: Set ${x_{\ds,v}:=1}$ for all ${v\in\mathcal{V}}$. For each ${v\in\mathcal{V}}$, let\linebreak
${d(v)=\left(\left(r_1,t_1\right),\dots,\left(r_{k_v},t_{k_v}\right)\right)}$ and ${t_0:= v}$. For ${i\in\left[k_v\right]}$, set ${x_{t_{i-1},t_i}:=1}$ and\linebreak
${z_{t_{i-1},r_i,t_i}:=1}$ if ${r_i\neq\emptyset}$. If ${d(v)\neq\emptyset}$ set ${x_{t_{k_v},\de}:=1}$, else set ${x_{v,\de}:=1}$. Set ${e_v:=f^0_v}$ for ${v\in\mathcal{V}}$ and $e_t$ as the respective fuel states after fulfilling trip ${t\in\Cupdot_{v\in\mathcal{V}} d(v)}$. For each ${c\in\mathcal{C}}$, set ${u_{m(c)}:=1}$. All other variables that have not been mentioned, are set to~$0$. The proof of the feasibility works analogously to the first part of the proof.

The objective function of the $\eqref{eq:MMILP}$ is constructed as follows: The vehicle cost $\cv$ arises for each non-empty duty. The deadhead cost between two subsequently fulfilled trips $s$ and $t$ is $\cd_{s,t}$ or ${\cd_{s,r}+\cd_{r,t}}$ if refuel point ${r\in\Rst}$ is visited in-between. The trip and route costs are charged, iff the respective trip or route is fulfilled. Therefore holds
\begin{align*}
	\operatorname{val}(S) = \operatorname{cost}\left(S'\right).
\end{align*}
%
\end{proof}

\begin{remark}

Consider that we transform a solution ${S=(x,z,e,u)}$ into $S'$ and transform $S'$ back to $S''$ as described in the proof. Note that the fuel states $e_t$ of the solutions $S$ and $S''$ do not necessarily coincide. This is caused by the construction of the $\eqref{eq:MMILP}$. The values for $x_{s,t}$, $z_{s,r,t}$ and $u_m$ coincide and therefore the objective function is not affected.

\end{remark}