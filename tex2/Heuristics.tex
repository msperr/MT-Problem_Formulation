\section{Heuristics}

There is already a heuristic for solving an easier version of the problem (Knoll, cap. 10). This heuristic only handles the case without costumers. This means, there is a trip set $\mathcal{T}$ and each of these trips has to be fulfilled. This is even a simplification to $\eqref{eq:Knoll:MILP}$. We try to extend this heuristic such that it can tackle the problem considering multimodal transport. 

\subsection{Splitting the Problem}

We define the splitting of the task graph similarly to (Knoll, cap. 8) with the difference, that the splittings can be defined generally here.

\begin{definition}[Splitting]
\label{def:splitting}

Let $n\in\mathbb{N}$ and let

\begin{align*}
	\Tcar=\Cupdot_{i=1}^n\Ti && \mathcal{V}=\Cupdot_{i=1}^n\Vi
\end{align*}

be partitions of the set of car trips, respectively vehicles. Then we call $\left\{\Ti|i\in[n]\right\}$ and $\left\{\Vi|i\in[n]\right\}$ splitting of $\mathcal{T}$ and $\mathcal{V}$ and $\Ti$ and $\Vi$ partial trip respectively vehicle set.

\end{definition}

\begin{definition}[Transformed Task Graph]

Let $\left\{\mathcal{T}_1,\dots\mathcal{T}_n\right\}$ be a splitting of $\Tcar$ according to Definition \ref{def:splitting}. Then we define:

\begin{enumerate}
	\item{Split Point: Let $s\in\Ti$ for $i\in[n]\backslash\{1\}$. For $j\in[i-1]$, we define the split point $\SPjs$ with $\pstart_{\SPjs}=\pend_{\SPjs}=:\pstart_s, \zstart_{\SPjs}=\zend_{\SPjs}=:\zstart_s$ and $\ft_{\SPjs}=:0$.}
	\item{For $i\in[n]\backslash\{1\}$ and $j\in[i-1]$, we define $P_{j,i}:=\left\{\SPjs|s\in\Ti\right\}$.}
	\item{Partial Split Point Set: For $j\in[n-1]$, we define the partial split point set $P_j:=\Cupdot_{i=j+1}^n P_{j,i}$.}
	\item{Split Point Set: We define the split point set $P:=\Cupdot_{j=1}^{n-1}P_j$.}
\end{enumerate}

Let $G=(V,A)$ the task graph, $\left\{\mathcal{V}_1,\dots,\mathcal{V}_n\right\}$ be a splitting of $\mathcal{V}$.

\begin{enumerate}
	\item{Transformed Task Graph: We define the transformed task graph $\overline{G}=\left(\overline{V},\overline{A}\right)$ with vertex set
		\begin{align*}
			\overline{V} := V\cup P = V\cup\left\{\operatorname{SP}_i(s)|i\in[n-1],j\in[n+1]\backslash[i],s\in\mathcal{T}_j\right\}
		\end{align*}
		and arc set
		\begin{align*}
			\overline{A} := & \left(\ds\times\mathcal{V}\right)\cupdot\bigcup_{i=1}^n\left\{(s,t)\in\left(\Vi\cupdot\Ti\right)\times\left(\Ti\cup P_i\right)|s\prec t\right\} \\
			& \cupdot\bigcup_{i=1}^n\left\{(s,t)\in\left(\left(\bigcup_{j=1}^{i-1} P_{j,i}\right)\times\Ti\right)|s=\operatorname{SP}_i(t)\right\}\cupdot \left(\mathcal{V}\times\left\{\de\right\}\right)\cupdot\left(\Tcar\times\left\{\de\right\}\right)
		\end{align*}}
\end{enumerate}

\end{definition}

\begin{definition}[Partial Transformed Task Graph]

Let $i\in[n]$. For a set of start points $\Vhat_i$, a set of end points $\Phat_i$ and the partial trip set $\Ti$, the partial transformed task graph is the directed graph $\overline{G}_i=\left(\overline{A}_i,\overline{V}_i\right)$ with vertex set

\begin{align*}
	\overline{V}_i := \left\{\ds,\de\right\}\cupdot\Vhat_i\cupdot\Ti\cupdot\Phat_i
\end{align*}

and arc set

\begin{align*}
	\overline{A}_i := & \left(\left\{\ds\right\}\times\left(\Vhat_i\cupdot\Ti\cupdot\Phat_i\right)\right)\cupdot\left\{(s,t)\in\left(\Vhat_i\cupdot\Tcar\right)\times\left(\Ti\cupdot\Phat_i\right)|s\preceq t\right\} \\
	& \cupdot\left(\left(\Vhat_i\cupdot\Ti\cupdot\Phat_i\right)\times\left\{\de\right\}\right)
\end{align*}

\end{definition}

%---------------------------------------------------------------------------------------------------------------------------------------

\subsection{Costumer-dependent Splitting}

In contrast to the splitting performed in (Knoll, cap. 8), the trips are not split according to their start times but according to their costumers' start times. This means, that all trips of a route and all routes of a costumer are in the same splitting. For each spitting, we apply (EMILP) to receive an optimal partial solution and connect the partial solutions to a feasible overall solution.

\paragraph{Splitting} \parfill

Given points in time $c_i$ for $i\in[n-1]$ with $c_i<c_{i+1}$ for $i\in[n-2]$. We first define a splitting of the costumers $\mathcal{C}=\Cupdot_{i=1}^n\mathcal{C}_i$ as

\begin{align*}
	\mathcal{C}_i := \begin{cases}
		\left\{c\in\mathcal{C}|\zstart_c\leq c_1\right\} & \text{for } i=1 \\
		\left\{c\in\mathcal{C}|c_{i-1}<\zstart_c\leq c_i\right\} & \text{for } i\in[n-1]\backslash\{1\} \\
		\left\{c\in\mathcal{C}|c_{n-1}<\zstart_c\right\} & \text{for } i=n.
	\end{cases}
\end{align*}

Based on the costumer splitting, we define the splittings of $\Tcar$ and $\mathcal{V}$ as

\begin{align*}
	\Ti := \left\{t\in\Tcar|(M\circ C)(t)\in\mathcal{C}_i\right\} && \text{for } i\in[n]
\end{align*}

and

\begin{align*}
	\Vi := \begin{cases}
		\left\{v\in\mathcal{V}|z_v\leq c_1\right\} & \text{for } i=1 \\
		\left\{v\in\mathcal{V}|c_{i-1}<z_v\leq c_i\right\} & \text{for } i\in[n-1]\backslash\{1\} \\
		\left\{v\in\mathcal{V}|c_{n-1}<z_v\right\} & \text{for } i=n.
	\end{cases}
\end{align*}

\paragraph{Solving of the Partial Instances} \parfill

Since for costumer $c\in\mathcal{C}_i$ all his trips are in splitting $\Ti$, costumer $c$ has to be satisfied only in the partial instance~$i$. For solving the partial instances, we modify the $(\operatorname{PLMILP}_i)$ from (Knoll, cap. 10) as follows: The constraint 

\begin{align}
	\sum_{s\in\operatorname{N}_{\overline{G}_i}^-(t)} x_{s,t} = 1 &&& \text{for all } t\in\hat{\mathcal{V}}_i\cup\Ti\cup\hat{P}_i \tag{10.1}
\end{align}

ensures that each trip in this partial instance is fulfilled. This constraint is replaced by

\begin{align}
	& \sum_{m\in C^{-1}(c)} u_m = 1 && \text{for all } c\in\mathcal{C}_i \label{eq:EMILP:costumer} \\
	&\sum_{s\in\operatorname{N}_{\overline{G}_i}^-(t)} x_{s,t} = u_m && \text{for all } m\in C^{-1}\left(\mathcal{C}_i\right), t\in m \label{eq:EMILP:route} \\
	&\sum_{s\in\operatorname{N}_{\overline{G}_i}^-(t)} x_{s,t} = 1 && \text{for all } t\in\hat{\mathcal{V}}_i\cup\hat{P}_i \label{eq:EMILP:pv}
\end{align}

where $\eqref{eq:EMILP:pv}$ ensures that all vehicles and split points are considered. $\eqref{eq:EMILP:costumer}$ and $\eqref{eq:EMILP:route}$ guarantee that for every costumer in this partial instance exactly one route is chosen.

The further procedure is similar to (Knoll, Cap. 10). For each solved partial instance~$i$, the partial split point set $\hat{P}_i$ is created. Therefore, only the chosen trips are considered. All trips that were not chosen in the partial instance, are neglected. The partial solutions are feasibly connected to a feasible overall solution according to (Knoll, cap. 10.2).

\begin{align}
\label{eq:PCMILP}
	\min & \omit\rlap{$\displaystyle{\left(\sum_{s\in\Ti\cup\Phat_i} x_{\ds,s} - \sum_{s\in\Phat_i} x_{s,\de}\right)\cv + \sum_{m\in C^{-1}\left(\mathcal{C}_i\right)} u_m \croute_m}$} \nonumber \\
	& \omit\rlap{$\displaystyle{\sum_{t\in\Ti\cup\Phat_i}\sum_{s\in\Ninoi(t)\backslash\left\{\de\right\}}\left[x_{s,t}\left(\cd_{s,t}+\ct_t\right)+\sum_{r\in\Rst}z_{s,r,t}\left(\cd_{s,r}+\cd_{r,t}-\cd_{s,t}\right)\right]}$} \tag{$\operatorname{PCMILP}_i$} \\
	\text{s.t.} & \sum_{t\in\Ninoi(s)} x_{t,s} = \sum_{t\in\Noutoi(s)} x_{s,t} & & \text{for all } s\in \overline{V}_i\backslash\left\{\ds,\de\right\} \\
	& \sum_{m\in C^{-1}(c)} u_m = 1 && \text{for all } c\in\mathcal{C}_i \tag{\ref{eq:EMILP:costumer}} \\
	& \sum_{s\in\Ninoi(t)} x_{s,t} = u_m && \text{for all } m\in C^{-1}\left(\mathcal{C}_i\right), t\in m \tag{\ref{eq:EMILP:route}} \\
	& \sum_{s\in\Ninoi(t)} x_{s,t} = 1 && \text{for all } t\in\Vhat_i\cup\Phat_i \tag{\ref{eq:EMILP:pv}} \\
	& \sum_{r\in\Rst} z_{s,r,t} \leq x_{s,t} & & \text{for all } t\in\Ti\cup\Phat_i, s\in\Ninoi(t)\backslash\Phat_i \\
	& e_s \leq f_s^0 & & \text{for all } s\in\Vhat_i \\
	& f^0_s \leq e_s && \text{for all } s\in\Phat_i \tag{10.2} \\
	& 0 \leq e_s - \sum_{r\in\Rst} z_{s,r,t}\fd_{s,r} & & \text{for all } t\in\Ti\cup\Phat_i, s\in\Ninoi(t)\backslash\Phat_i \\
	& e_t \leq 1 - \ft_t - \sum_{r\in\Rst} z_{s,r,t}\fd_{r,t} & & \text{for all } t\in\Ti\cup\Phat_i, s\in\Ninoi(t)\backslash\Phat_i \\
	& \omit\rlap{$\displaystyle{e_t \leq e_s - x_{s,t}\left(f_{s,t}^{\operatorname{d}}+f_t^{\operatorname{t}}\right) - \sum_{r\in\Rst} z_{s,r,t}\left(\fd_{s,r}+\ft_r+\fd_{r,t}-\fd_{s,t}\right) + \left(1-x_{s,t}\right)}$} \nonumber \\
	& & & \text{for all } t\in\Ti\cup\Phat_i, s\in\Ninoi(t)\backslash\Phat_i \\
	& e_t \leq e_s-x_{s,t}\ft_t+\left(1-x_{s,t}\right) && \text{for all } s\in\Phat_i,t\in\Noutoi(s) \\
	& e_s + \ft_s \leq f^{\max}_s + \left(1-x_{\ds,s}\right)\cdot\left(1+\ft_s\right) && \text{for all } s\in\Ti \tag{10.3} \\
	& f^{\min}_s \leq e_s + \left(1-x_{s,\de}\right) && \text{for all } s\in\Ti \tag{10.4} \\
	& x_{s,t}\in\{0,1\} & & \text{for all } (s,t)\in\overline{A}_i \\
	& z_{s,r,t}\in\{0,1\} & & \text{for all } t\in\Ti\cup\Phat_i,s\in\Ninoi(t)\backslash\Phat_i,r\in\Rst \\
	& e_s\in[0,1] & & \text{for all } s\in\overline{V}_i\backslash\left\{\ds,\de\right\} \\
	& u_m\in\{0,1\} && \text{for all } m\in C^{-1}\left(\mathcal{C}_i\right)
\end{align}

\paragraph{Model Equivalence} \parfill

This heuristic formulation is not equivalent to the original formulation $\eqref{eq:MMILP}$. This is shown by the following example.

\begin{example}

Let $t_1$, $t_2$, $t_3$ with $t_1\prec t_2\prec t_3$ be trips with the properties shown in Table \ref{tab:costumer_heuristic_example}

\begin{table}[hb]
	\centering
	\begin{tabular}{c|cccc}
		Trip & Start & End & Route & Costumer \\
		\hline
		$t_1$ & 8:00 & 8:15 & $m_1$ & $C_1$ \\
		$t_2$ & 8:30 & 8:45 & $m_2$ & $C_2$ \\
		$t_3$ & 9:00 & 9:15 & $m_1$ & $C_1$ \\
	\end{tabular}
	\caption{Trips}
	\label{tab:costumer_heuristic_example}
\end{table}

In this case, costumer $C_1$ uses public transport between 8:15 and 9:00. The duty $\left(t_1,t_2,t_3\right)$ is a feasible result of the $\eqref{eq:MMILP}$.

If there is a split point at 8:15 then the splittings are $\mathcal{T}_1=\left\{t_1,t_3\right\},\mathcal{T}_2=\left\{t_2\right\}$. Hence, there is one split point $\operatorname{SP}_1\left(t_2\right)$ with $\zstart_{\operatorname{SP}_1\left(t_2\right)} =$ 8:30. The partial solution of instance~$1$ is $\left(t_1,t_3\right)$ and $t_3\not\prec \operatorname{SP}_1\left(t_2\right)$. Thus, the partial solutions cannot be feasibly connected to the solution $\left(t_1,t_2,t_3\right)$.

\end{example}

With this example we have seen, that the formulations (EMILP) and $\eqref{eq:MMILP}$ are not equivalent. It is even possible, that an optimal solution of $\eqref{eq:MMILP}$ is not feasible in (EMILP).

Although the formulations are not equivalent, we can give an estimation on the objective value when we make some restrictions.

\begin{definition}

For $n\geq 3$, consider a costumer set $\mathcal{C}$ and split points $c_i$ for $i\in[n-1]$ with $c_i<c_{i+1}$ for all $i\in[n-2]$. We define the following values:

\begin{itemize}
	\item{Costumer Extension for $c\in\mathcal{C}$: $\displaystyle{L_{\operatorname{C}}(c) := \max_{t\in(M\circ C)^{-1}(c)}\zstart_t - \min_{t\in(M\circ C)^{-1}(c)}\zstart_t}$}
	\item{Costumer Extension: $\displaystyle{L_{\operatorname{C}} := \max_{c\in\mathcal{C}} L_{\operatorname{C}}(c)}$}
	\item{Splitting Length: $\displaystyle{L_{\operatorname{S}} := \min_{i\in[n-1]} c_{i+1}-c_i}$}
\end{itemize}

\end{definition}

\begin{theorem}

For $n\geq 3$, consider the problem with costumer set $\mathcal{C}$ and split points $c_i$ for $i\in[n-1]$ with $c_i<c_{i+1}$ for all $i\in[n-2]$. Let

\begin{align}
\label{eq:LCLS}
	L_{\operatorname{C}}\leq L_{\operatorname{S}}
\end{align}

Let $d:=\left(t_1,\dots,t_k\right)$ be the duty of a vehicle of a feasible solution of the $\eqref{eq:MMILP}$. Then, there are duties $d_1\cupdot d_2=d$, where $d_1,d_2$ are part of a feasible solution of (EMILP). Moreover, there holds

\begin{align}
	\operatorname{cost}\left(d_1\right)+\operatorname{cost}\left(d_2\right)\leq 2\cdot\operatorname{cost}\left(d\right).
\end{align}

\end{theorem}

\begin{proof}

We consider the vehicle duty $d=\left(t_1,\dots t_k\right)$. We write $s\prec t$ according to Definition~$\ref{def:partial_order}$, i.e. $(s,t)$ is feasible in $\eqref{eq:MMILP}$. We write $s\to t$ iff $(s,t)$ is feasible in (EMILP).

Consider $s\prec t$ with $s\not\to t$ and costumers $C_s:=(M\circ C)(s)$ and $C_t:=(M\circ C)(t)$. Then $s$ is in a later splitting than $t$. There are split points $c_{l-1},c_l,c_{l+1}$ for $l\in[n]$ with

\begin{align*}
	\zstart_s < \zstart_t && \zstart_{C_t}\leq c_l <\zstart_{C_s} && c_l+L_{\operatorname{S}}\leq c_{l+1} && \zstart_{C_s}\leq \zstart_s\leq \zstart_{C_s}+L_{\operatorname{C}}
\end{align*}

Since $\eqref{eq:LCLS}$, holds

\begin{gather*}
	\zstart_{C_s} \leq \zstart_s < \zstart_t\leq\zstart_{C_t}+L_{\operatorname{C}} \leq c_l+L_{\operatorname{C}} \leq c_l+L_{\operatorname{S}} \leq c_{l+1} \\
	\zstart_{C_t} \geq \zstart_t-L_{\operatorname{C}} > \zstart_s-L_{\operatorname{C}} \geq \zstart_{C_s}-L_{\operatorname{C}} > c_l-L_{\operatorname{C}}\geq c_l-L_{\operatorname{S}}\geq c_{l-1}
\end{gather*}

and therefore $t\in\mathcal{T}_l,s\in\mathcal{T}_{l+1}$. Here, we use $c_{0}:=-\infty,c_{n+1}:=+\infty$.

\paragraph{Feasibility}

For arbitrary $i\in[k-2]$ holds: $t_i\prec t_{i+1}\prec t_{i+2}$, therefore also $t_i\prec t_{i+2}$. We prove that $t_{i+2}$ can be appended after $t_{i}$ or $t_{i+1}$. We differentiate between the following cases:

\begin{enumerate}
	\item{$t_{i+1}\to t_{i+2}$:}
		Clear.
	\item{$t_{i+1}\not\to t_{i+2}$:}
		Then holds $t_{i+2}\in\mathcal{T}_l$ and $t_{i+1}\in\mathcal{T}_{l+1}$ for some $l\in[k]$. From $t_i\prec t_{i+2}$ follows $t_i\in\bigcup_{j=1}^{l+1}\mathcal{T}_j$. Therefore $t_i\to t_{i+1}$ or $t_i\to t_{i+2}$.
	\begin{itemize}
		\item{$t_i\to t_{i+2}$:}
			Clear.
		\item{$t_i\not\to t_{i+2}$:}
			Then holds $t_{i+2}\in\mathcal{T}_l$ and $t_i,t_{i+1}\in\mathcal{T}_{l+1}$ and therefore $t_i\to t_{i+1}$. For $i'\geq i$ holds $t_{i'}\in\bigcup_{j=l}^n \mathcal{T}_j$ and therefore $t_{i+1}\to t_{i'}$ or $t_{i+2}\to t_{i'}$. Thus, every later trip can be appended after on of these duties.
	\end{itemize}
\end{enumerate}

We have seen that two duties $d_1,d_2$ can fulfill the trips of duty $d$, such that $d_1$ and $d_2$ are feasible in (EMILP). Each trip can be appended to $d_1$ or to $d_2$.

\paragraph{Costs}

The costs of duty $d$ are

\begin{align*}
	\operatorname{cost}(d)=\cv+\cd_{v,t_1}+\ct_{t_1}+\sum_{i=2}^k \left(\cd_{t_{i-1},t_i}+\ct_{t_i}\right).
\end{align*}

Each duty $d_1,d_2$ has cost $\cd_{t,t'}+\ct_{t'}+\cd_{t',t''}$ if trip $t'$ is covered and cost $\cd_{t,t''}$ if not. According to $\eqref{eq:triangle_inequality_ext}$, the costs for not covering the trip do not exceed the costs for covering. Therefore we have

\begin{align*}
	\operatorname{cost}\left(d_1\right)+\operatorname{cost}\left(d_2\right)\leq 2\cdot\operatorname{cost}\left(d\right).
\end{align*}

\end{proof}

\begin{corollary}

Consider the problem with $L_{\operatorname{C}}\leq L_{\operatorname{S}}$. Let $S_1$ be a feasible solution of $\eqref{eq:MMILP}$. Then there exists a solution $S_2$ feasible also in (EMILP) such that

\begin{align*}
	\operatorname{val}\left(S_2\right)\leq 2\cdot\operatorname{val}\left(S_1\right)
\end{align*}

\end{corollary}

%----------------------------------------------------------------------------------------------------------------------------------------

\subsection{Time-dependent Splitting}

The developed formulation (EMILP) based on a costumer-dependent splitting is not equivalent to the original formulation $\eqref{eq:MMILP}$. The goal now is to develop a splitting that is equivalent and create a heuristic based on this splitting. Therefore, it is necessary that a splitting of trips of the same costumer is possible. This leads to the following problem: When the partial instances are solved successively, we need a possibility to still guarantee the costumer satisfaction for the entire problem. This has to be applied already in the partial instance, although we do not have any knowledge about the trips of the same costumer in the later solved partial instances.

\subsubsection{Basic Idea}
\label{subsubsec:basic_idea}

\paragraph{Splitting} \parfill

Given points in time $c_i$ for $i\in[n]$ with $c_i<c_{i+1}$ for $i\in[n-1]$. We define the splitting of $\Tcar$ and $\mathcal{V}$ as follows:

\begin{align*}
	\Ti := \begin{cases}
		\left\{t\in\Tcar|\zstart_t\leq c_1\right\} & \text{for } i=1 \\
		\left\{t\in\Tcar|c_{i-1}<\zstart_t\leq c_i\right\} & \text{for } i\in[n]\backslash\{1\} \\
		\left\{t\in\Tcar|c_n<\zstart_t\right\} & \text{for } i=n+1.
	\end{cases}
\end{align*}

and

\begin{align*}
	\Vi := \begin{cases}
		\left\{v\in\mathcal{V}|z_v\leq c_1\right\} & \text{for } i=1 \\
		\left\{v\in\mathcal{V}|c_{i-1}<z_v\leq c_i\right\} & \text{for } i\in[n]\backslash\{1\} \\
		\left\{v\in\mathcal{V}|c_n<z_v\right\} & \text{for } i=n+1.
	\end{cases}
\end{align*}

\paragraph{Solving of the Partial Instances} \parfill

Let $[n]$ be the set of partial instances, let $\sigma\in S_n$ with $\sigma(n)=1$ be the order, in which the partial instances are processed. This means, partial instance $\sigma(i)$ is solved at the $i$-th position, the first partial instance is always solved at last. The actual description of $\sigma$ follows later.

Since the trips of the same costumer may be in different splittings, we cannot easily guarantee the costumer satisfaction only in one partial instance. We have to put great effort in this issue. For this, we first define the earliest partial instance in which a trip of a costumer arises as follows:

\begin{align*}
	\gamma: \mathcal{C}\to[n] && \gamma(c):=\argmin_{i\in[n]}\left\{\sigma(i)\in[n]|\left((M\circ C)^{-1}(c)\cap\Ti\right)\neq\emptyset\right\}
\end{align*}

Depending on $\gamma$ and $\left\{\mathcal{T}_1,\dots\mathcal{T}_n\right\}$ we define a partition $\mathcal{C}=\left\{\mathcal{C}_1,\dots\mathcal{C}_n\right\}$ as

\begin{align}
	\mathcal{C}_i := \left\{c\in\mathcal{C}|\gamma(c)=i\right\} && \text{for } i\in[n]
\end{align}

Consider costumer $c\in\mathcal{C}$ and the partial instance $\gamma(c)\in[n]$. In this partial instance, a multimodal route $m\in C^{-1}(c)$ for the costumer is chosen and this choice is definite. This means, in all subsequent partial instances, all trips $t\in m$ are fixed to be chosen before solving and all trips $t\in(M\circ C)^{-1}(c)\backslash m$ are fixed to be neglected.

In partial instance $\gamma(c)$ we have at least one trip of this costumer. But there are also trips that are in other splittings. There are even multimodal routes with no trip in this splitting at all. These routes must not be neglected. Therefore, we need a method to choose the routes where all routes $m\in C^{-1}(c)$ are considered. Therefore, we try to estimate the costs of the routes in advance.

The solving of the partial instances is again based on the formulation $(\operatorname{PLMILP}_i)$ from (Knoll, cap. 10). The constraint

\begin{align}
	\sum_{s\in\operatorname{N}_{\overline{G}_i}^-(t)} x_{s,t} = 1 &&& \text{for all } t\in\hat{\mathcal{V}}_i\cup\Ti\cup\hat{P}_i \tag{10.1}
\end{align}

is removed. Instead, this is replaced by

\begin{align}
	&\sum_{m\in C^{-1}(c)} u_m = 1 && \text{for all } c\in\mathcal{C}_i \label{eq:TMILP:costumer} \\
	&\sum_{s\in\operatorname{N}_{\overline{G}_i}^-(t)} x_{s,t} = u_m && \text{for all } m\in C^{-1}\left(\mathcal{C}_i\right), t\in m\cap\Ti \label{eq:TMILP:route} \\
	&\sum_{s\in\operatorname{N}_{\overline{G}_i}^-(t)} x_{s,t} = 1 && \text{for all } t\in\hat{\mathcal{V}}_i\cup\hat{P}_i \label{eq:TMILP:pv}.
\end{align}

Remember, that $\mathcal{C}_i$ describes the splitting according to the first appearance of trips of costumer $c$ in the algorithm. $\eqref{eq:TMILP:costumer}$ and $\eqref{eq:TMILP:route}$ ensure that for every costumer in this partial instance exactly one route is chosen. In this constraint it is irrelevant, if there are trips of the considered routes in this splitting. 

After solving the partial instance, all determined $u_m$ are fixed for the later processed partial instances. The fixed route decisions from the previous partial instances have an impact on the instance, too. 

Let $\bar{u}_m\in[0,1]$ be the fixed route choices from the previous instances. Define

\begin{align}
	\overline{\mathcal{C}}_i := \left\{c\in\mathcal{C}|\gamma\left(c\right)<\sigma(i)\right\}
\end{align}

as the costumers that are already treated. Then, we introduce the constraint

\begin{align}
	&\sum_{s\in\operatorname{N}_{\overline{G}_i}^-(t)} x_{s,t} = \bar{u}_m && \text{for all } m\in C^{-1}\left(\overline{\mathcal{C}}_i\right), t\in m\cap\Ti \label{eq:TMILP:route_fix}
\end{align}

which ensures that the previous route choices are considered.

\paragraph{Cost Estimation} \parfill

In order to choose a route in a partial instance, we have to estimate the costs for this routes in advance in all subsequent instances. The entire cost for the problem consists of vehicle costs $\cv$, trip costs $\ct$, deadhead costs $\cd$ and route costs $\hat{c}^{\operatorname{r}}$. While we can determine the trip and route costs easily for a route, the vehicle costs and trip costs strongly depend on the environment of the route and cannot be determined. We therefore focus on the trip and route costs and define the estimated route costs as follows:

\begin{align}
\label{eq:estimated_route}
	C_1(m) := \croute_m + \sum_{t\in m}\ct_t && \text{for } m\in\mathcal{M}
\end{align}

We use these costs in the $(\operatorname{TMILP}_i)$ to define the route costs

\begin{align}
	\hat{c}^{\operatorname{r}}_m := \hat{c}^{\operatorname{r}}_m + \sum_{t\in m\backslash\Ti}\ct_t && \text{for } m\in\mathcal{M}
\end{align}

and add 

\begin{align*}
	\sum_{m\in C^{-1}\left(\mathcal{C}_i\right)} u_m \hat{c}^{\operatorname{r}}_m
\end{align*}

to the objective function.

In $\hat{c}^{\operatorname{r}}_m$, the trips in the same splitting are not considered since they are already part of the objective function. The trips that are not in the same splitting are added to $\hat{c}^{\operatorname{r}}_m$, such that they have an impact on the choice of the routes.

Consider a trip $t$ that is decided before this partial instance, i.e. $t\in(M\circ C)\left(\overline{\mathcal{C}}_i\right)$. Its trip costs $\ct_t$ arise twice in the objective functions. Once in the partial instance $\gamma\left((M\circ C)(t)\right)$ as part of $\hat{c}^{\operatorname{r}}_{M(t)}$ and once in partial instance $i$ as $\ct_t$. But since in partial instance $i$, the trip has fulfilled anyway, these costs are only an additional factor that does not influence the solution.

\paragraph{LP Formulation} \parfill

For partial instance $i\in[n]\backslash\{1\}$, $\eqref{eq:PTMILP}$ is solved:

\newpage
\pagestyle{empty}

\begin{align}
\label{eq:PTMILP}
	\min & \omit\rlap{$\displaystyle{\left(\sum_{s\in\Ti\cup\Phat_i} x_{\ds,s} - \sum_{s\in\Phat_i} x_{s,\de}\right) \cv + \sum_{m\in C^{-1}\left(\mathcal{C}_i\right)} u_m \hat{c}^{\operatorname{r}}_m}$} \nonumber \\
	& \omit\rlap{$\displaystyle{\sum_{t\in\Ti\cup\Phat_i}\sum_{s\in\Ninoi(t)\backslash\left\{\de\right\}}\left[x_{s,t}\left(\cd_{s,t}+\ct_t\right)+\sum_{r\in\Rst}z_{s,r,t}\left(\cd_{s,r}+\cd_{r,t}-\cd_{s,t}\right)\right]}$} \tag{$\operatorname{PTMILP}_i$} \\
	\text{s.t.} & \sum_{t\in\Ninoi(s)} x_{t,s} = \sum_{t\in\Noutoi(s)} x_{s,t} & & \text{for all } s\in \overline{V}_i\backslash\left\{\ds,\de\right\} \\
	&\sum_{m\in C^{-1}(c)} u_m = 1 && \text{for all } c\in\mathcal{C}_i \tag{\ref{eq:TMILP:costumer}} \\
	&\sum_{s\in\Ninoi(t)} x_{s,t} = u_m && \text{for all } m\in C^{-1}\left(\mathcal{C}_i\right), t\in m\cap\Ti \tag{\ref{eq:TMILP:route}} \\
	&\sum_{s\in\Ninoi(t)} x_{s,t} = 1 && \text{for all } t\in\hat{\mathcal{V}}_i\cup\hat{P}_i \tag{\ref{eq:TMILP:pv}} \\
	&\sum_{s\in\Ninoi(t)} x_{s,t} = \bar{u}_m && \text{for all } m\in C^{-1}\left(\overline{\mathcal{C}}_i\right), t\in m\cap\Ti \tag{\ref{eq:TMILP:route_fix}} \\
	& \sum_{r\in\Rst} z_{s,r,t} \leq x_{s,t} & & \text{for all } t\in\Ti\cup\Phat_i, s\in\Ninoi(t)\backslash\Phat_i \\
	& e_s \leq f_s^0 & & \text{for all } s\in\Vhat_i \\
	& f^0_s \leq e_s && \text{for all } s\in\Phat_i \tag{10.2} \\
	& 0 \leq e_s - \sum_{r\in\Rst} z_{s,r,t}\fd_{s,r} & & \text{for all } t\in\Ti\cup\Phat_i, s\in\Ninoi(t)\backslash\Phat_i \\
	& e_t \leq 1 - \ft_t - \sum_{r\in\Rst} z_{s,r,t}\fd_{r,t} & & \text{for all } t\in\Ti\cup\Phat_i, s\in\Ninoi(t)\backslash\Phat_i \\
	& \omit\rlap{$\displaystyle{e_t \leq e_s - x_{s,t}\left(f_{s,t}^{\operatorname{d}}+f_t^{\operatorname{t}}\right) - \sum_{r\in\Rst} z_{s,r,t}\left(\fd_{s,r}+\ft_r+\fd_{r,t}-\fd_{s,t}\right) + \left(1-x_{s,t}\right)}$} \nonumber \\
	& & & \text{for all } t\in\Ti\cup\Phat_i, s\in\Ninoi(t)\backslash\Phat_i \\
	& e_t \leq e_s-x_{s,t}\ft_t+\left(1-x_{s,t}\right) && \text{for all } s\in\Phat_i,t\in\Noutoi(s) \\
	& e_s + \ft_s \leq f^{\max}_s + \left(1-x_{\ds,s}\right)\cdot\left(1+\ft_s\right) && \text{for all } s\in\Ti \tag{10.3} \\
	& f^{\min}_s \leq e_s + \left(1-x_{s,\de}\right) && \text{for all } s\in\Ti \tag{10.4} \\
	& x_{s,t}\in\{0,1\} & & \text{for all } (s,t)\in\overline{A}_i \\
	& z_{s,r,t}\in\{0,1\} & & \text{for all } t\in\Ti\cup\Phat_i,s\in\Ninoi(t)\backslash\Phat_i,r\in\Rst \\
	& e_s\in[0,1] & & \text{for all } s\in\overline{V}_i\backslash\left\{\ds,\de\right\} \\
	& u_m\in\{0,1\} && \text{for all } m\in C^{-1}\left(\mathcal{C}_i\right)
\end{align}

\pagestyle{headings}
\newpage

%----------------------------------------------------------------------------------------------------------------------------------------

\subsubsection{Iterative Approach}

We use the previously developed heuristic for an iterative approach. We compute an initial solution while we choose the routes with cost function $C_1$. Then we determine the actual cost of this route in the entire solution and compare the estimated cost with the actual cost.

\paragraph{Initial Solution} \parfill

We determine a solution with the heuristic developed in section \ref{subsubsec:basic_idea}. Given a solution $S=\left(\bar{x},\bar{z},\bar{e},\bar{u}\right)$ of the (TMILP), we determine 

\begin{align*}
	C_1(c) := C_1(m) && \text{for } c\in\mathcal{C}, m\in C^{-1}(c) \text{ with } \bar{u}_m = 1
\end{align*}

\paragraph{Subproblem} \parfill

Given a solution of the problem, the subproblem is to find a costumer with a bad route choice. This means, for this costumer there is another route, such that the total costs are lower if this route is chosen. Then, we can exchange these routes and compute a new solution considering the new route. 

An initial idea is to compute the costs, one route in the solution contributes to the entire solution. Then, we can compare this to the cost, with which we estimated the route costs before. If the actual costs are considerably higher than the estimate costs, this costumer is a candidate for exchanging routes.

Since we cannot determine the contributing costs exactly, we try to estimate these costs.

Let $S=\left(\bar{x},\bar{z},\bar{e},\bar{u}\right)$ be a solution of the (TMILP). To determine the contributing costs for route $m\in\mathcal{M}$, we define the following auxiliary costs for every trip $t\in\mathcal{T}$ of the solution:

Vehicle costs $\cv_t(S)$: Let $v\in\mathcal{V}$ be the vehicle covering $t$ and $k_v$ the number of trips covered by $v$:
\begin{align*}
	\cv_t(S) := \frac{\cv}{k_v}
\end{align*}

Refueling costs $c^{\operatorname{refuel}}_t(S)$: Let $r\in\mathcal{R}$ be the next refuel station used after $t$ and $T_r$ all trips covered since the last station, let $\bar{z}_{s,r,s'} = 1$:
\begin{align*}
	c^{\operatorname{refuel}}_t(S) := \frac{\ft_t}{\sum_{t'\in T_r} \ft_t}\left(\cd_{s,r}+\cd_{r,s'}-\cd_{s,s'}\right)
\end{align*}
If the vehicle is not refueled after $t$, then $c^{\operatorname{refuel}}_t(S) := 0$.

Deadhead costs $\cd_t(S)$: Let $s\in\mathcal{V}\cup\Tcar,s'\in\Tcar$ be the trips covered directly before and after $t$ by vehicle $v$, i.e. $\bar{x}_{s,t}=\bar{x}_{t,s'}=1$:
\begin{align*}
	\cd_t(S) := \frac 1 2 \left(\cd_{s,t}+\cd_{t,s'}\right)
\end{align*}
If $t$ is the last trip of the duty, i.e. $\bar{x}_{s,t}=\bar{x}_{t,d^{\operatorname{e}}}=1$, then $\cd_t(S) := \frac 1 2 \cd_{s,t}$.

With these auxiliary costs we can define new route costs which describe the contribution of a multimodal route to the entire solution better:

\begin{definition}[Improved Cost Estimation]

Let $S=\left(\bar{x},\bar{z},\bar{e},\bar{u}\right)$ be a solution of the (TMILP). With the auxiliary costs described before, we define the improved cost estimation for all multimodal routes $\left\{m\in\mathcal{M}|\bar{u}_m=1\right\}$:

\begin{align*}
	C_2(S,m) := C_1(m) + \sum_{t\in m}\left(\cv_t(S) + c^{\operatorname{refuel}}_t(S) + \cd_t(S)\right)
\end{align*}

We further define

\begin{align*}
	C_2(S,c) := C_2(S,m) && \text{for } c\in\mathcal{C},m\in C^{-1}(c) \text{ with } \bar{u}_m = 1
\end{align*}

\end{definition}

Now we can evaluate our previous for the route contribution. If the value of $\frac{C_2(S,c)}{C_1(c)}$ is bad then the probability is high, the we made a bad route choice for costumer $c\in\mathcal{C}$.

%----------------------------------------------------------------------------------------------------------------------------------------
