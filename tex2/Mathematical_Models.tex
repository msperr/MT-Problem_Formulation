\section{Mathematical Models}

We introduce the mathematical model, with which the previously described problem is solved.

\subsection{Task Graph}

For tackling the problem, we introduce a task graph, on which the model is based. The graph is basically the same as in (Kaiser, Knoll, cap. 3.1) with the restriction, that only car trips $t\in\Tcar$ are considered.

\begin{definition}[Task Graph]

Let $\ds,\de$ be special vertices describing the source and sink of the vehicle flow. We define the task graph as $\hat{G}=(\hat{V},\hat{A})$, where

\begin{align*}
	\hat{V}:=\left\{\ds,\de\right\}\cupdot\mathcal{V}\cupdot\Tcar
\end{align*}

is the vertex set consisting of the source, the sink, the vehicle set $\mathcal{V}$ and the set of car trips $\Tcar$. The arc set is

\begin{align*}
	\hat{A} & :=\left(\left\{\ds\right\}\times\mathcal{V}\right)\cupdot\left\{(s,t)\in\left(\mathcal{V}\cupdot\Tcar\right)^2|s\prec t\right\}\cupdot\left(\left(\mathcal{V}\cupdot\Tcar\right)\times\left\{\de\right\}\right).
\end{align*}

\end{definition}

A vertex $s\in\mathcal{V}$ represents the initial state of a vehicle $s$ where it becomes available for the first time. Each $\ds -\de$-path in $\hat{G}$ is the duty of one vehicle, i.e. this vehicle fulfills the trips in the order given by the path. Hence, two trips are connected only if it is possible that one car fulfills both trips, i.e. the relation $\prec$ holds.

To consider the refuel stations, we introduce an extended task graph.

\begin{definition}[Extended Task Graph]

For every $s,t\in \mathcal{V}\cupdot\Tcar$ with $s\prec t$ we create a copy of $\left\{r\in\mathcal{R}|\zend_s+t_{s,r}+t_{r,t}\leq\zstart_t\right\}$ denoted by $\Rst$. This means, various copied sets are pairwise disjoint. We define the extended task graph $G=(V,A)$ with vertex set

\begin{align*}
	V:=\hat{V}\cupdot\Cupdot_{\substack{s,t\in\mathcal{V}\cupdot\Tcar \\ s\prec t}} \Rst
\end{align*}

and arc set

\begin{align*}
	A:=\hat{A}\cupdot\left\{(s,r)|s,t\in\mathcal{V}\cupdot\Tcar,s\prec t,r\in\Rst\right\}\cupdot\left\{(r,t)|s,t\in\mathcal{V}\cupdot\Tcar,s\prec t,r\in\Rst\right\}.
\end{align*}

In this graph, each feasible refuel station is considered for each feasible pair of trips. There is a method to reduce the size of $\Rst$ significantly without losing the optimal solution. This method is described in (Kaiser, Knoll, cap. 3.2.2). From now on, we will use $G=(V,A)$ with restricted $\Rst$.

\end{definition}

%---------------------------------------------------------------------------------------------------------------------------------------

\subsection{Arc Flow Formulation}

We develop a model for solving the problem via a flow of the cars. The multimodal routes are given in advance. The car trips are adjusted in such a way, that they fit to the public transport routes (in location and time). It is not possible to model each route as a trip because then the car availabilities are not considered.

We model a flow of the cars. The public transport trips work only as constraints for this flow. Constraints are the fulfilling of one multimodal route per costumer and the fuel constraints.

\paragraph{Basic Model} \parfill

We model the arc flow as an ILP. The formulation is basically built on $\eqref{eq:Knoll:MILP}$, described in (Kaiser, Knoll, cap. 3.2). We use the following decision variables:
\begin{itemize}
	\item{$x_{s,t}\in\{0,1\}$ for $(s,t)\in A$: indicates, whether trip $t\in\Tcar$ is fulfilled after $s\in \mathcal{V}\cupdot\Tcar$}
	\item{$z_{s,r,t}\in\{0,1\}$ for $t\in\Tcar,s\in\Nin(t),r\in\Rst$: indicates, whether refuel station $r\in\mathcal{R}$ is visited between $s$ and $t$}
	\item{$e_s\in[0,1]$ for $s\in V\backslash\left\{\ds,\de\right\}$: states the fuel of the respective car after fulfilling trip $s\in\Tcar$}
\end{itemize}

If $s\in\mathcal{V}$, then $x_{s,t}$ determines, whether trip $t$ is the first trip fulfilled by $s$ and $e_s$ is the initial fuel state $f^0_s$ of vehicle $s$.

Additionally to $\eqref{eq:Knoll:MILP}$ we introduce decision variables to determine the fulfilling of routes:

\begin{itemize}
	\item{$u_m\in\{0,1\}$ for $m\in\mathcal{M}$: indicates whether multimodal route $m$ is fulfilled}
\end{itemize}

The basic constraints are developed in detail in (Kaiser, Knoll, cap. 3.2) and not shown in detail here. The basic constraints are the flow conservation constraint, the constraint for considering every car and the constraints guaranteeing feasible fuel states all the time.

\paragraph{Costumer and Route Constraints} \parfill

In $\eqref{eq:Knoll:MILP}$, each costumer has a set of alternative trips and from this set, exactly one trips has to be fulfilled. This is modeled as follows:

\begin{align*}
	\sum_{t\in C^{-1}(c)} \sum_{s\in\Nin(t)} x_{s,t} = 1 &&& \text{for all } c\in\mathcal{C} \tag{\ref{eq:Knoll:costumer}}
\end{align*}

In contrast to $\eqref{eq:Knoll:MILP}$, here each costumer has a set of alternative routes consisting of trips and from this set, exactly one route has to be fulfilled. Therefore, we replace $\eqref{eq:Knoll:costumer}$ by the following formulation:

\begin{align}
	\sum_{m\in C^{-1}(c)} u_m = 1 &&& \text{for all } c\in\mathcal{C} \label{eq:MMILP:costumer} \\
	\sum_{s\in\Nin(t)} x_{s,t} = u_m &&& \text{for all } m\in\mathcal{M}, t\in m \label{eq:MMILP:route}
\end{align}

The constraint $\eqref{eq:MMILP:costumer}$ says, that for every costumer exactly one route is fulfilled. The constraint $\eqref{eq:MMILP:route}$ says, if a route is fulfilled then every trip of this route must be fulfilled.

\paragraph{Objective Function} \parfill

The objective function in $\eqref{eq:Knoll:MILP}$ is given by

\begin{align*}
	\sum_{s\in\mathcal{V}} \sum_{t\in\Nout(s)\backslash\{d^{\operatorname{e}}\}} x_{s,t}\cv	+ \sum_{t\in\mathcal{T}} \sum_{s\in\Nin(t)} \left[x_{s,t}\left(\cd_{s,t}+\ct_{t}\right) + \sum_{r\in\Rst} z_{s,r,t}\left(\cd_{s,r}+\cd_{r,t}-\cd_{s,t}\right)\right]
\end{align*}

considering the vehicle costs $\cv$, the trip costs $\ct_t$ for $t\in\Tcar$ and the deadhead costs $\cd$. What is missing, are the route-dependent costs $\croute$. Thus, we add

\begin{align*}
	\sum_{m\in\mathcal{M}} u_m \croute_m
\end{align*}

to the objective function.

\paragraph{LP Formulation} \parfill

Putting all this together, we get the following formulation, called $\eqref{eq:MMILP}$:

\begin{align}
	\min \quad & \sum_{s\in\mathcal{V}} \sum_{t\in\Nout(s)\backslash\{d^{\operatorname{e}}\}} x_{s,t}\cv + \sum_{m\in\mathcal{M}} u_m \croute_m \nonumber \\
	& \omit\rlap{$\displaystyle{ + \sum_{t\in\mathcal{T}} \sum_{s\in\Nin(t)} \left[x_{s,t}\left(\cd_{s,t}+\ct_{t}\right) + \sum_{r\in\Rst} z_{s,r,t}\left(\cd_{s,r}+\cd_{r,t}-\cd_{s,t}\right)\right]}$} \tag{MMILP} \label{eq:MMILP} \\
	\text{s.t.} \quad & \sum_{t\in\Nin(s)} x_{t,s} = \sum_{t\in\Nout(s)} x_{s,t} & & \text{for all } s\in V\backslash\left\{d^{\operatorname{s}},d^{\operatorname{e}}\right\} \tag{3.15} \\
	& \sum_{s\in\Nin(t)} x_{s,t} = 1 & & \text{for all } t\in\mathcal{V} \tag{3.16} \\
	& \sum_{m\in C^{-1}(c)} u_m = 1 && \text{for all } c\in\mathcal{C} \tag{\ref{eq:MMILP:costumer}} \\
	& \sum_{s\in\Nin(t)} x_{s,t} = u_m && \text{for all } m\in\mathcal{M}, t\in m \tag{\ref{eq:MMILP:route}} \\
	& \sum_{r\in\Rst} z_{s,r,t} \leq x_{s,t} & & \text{for all } t\in\mathcal{T}, s\in\Nin(t) \tag{3.18} \\
	& e_s \leq f_s^0 & & \text{for all } s\in\mathcal{V} \tag{3.19} \\
	& 0 \leq e_s - \sum_{r\in\Rst} z_{s,r,t}\fd_{s,r} & & \text{for all } t\in\mathcal{T}, s\in\Nin(t) \tag{3.12} \\
	& e_t \leq 1 - \ft_t - \sum_{r\in\Rst} z_{s,r,t}\fd_{r,t} & & \text{for all } t\in\mathcal{T}, s\in\Nin(t) \tag{3.13} \\
	& \omit\rlap{$\displaystyle{e_t \leq e_s - x_{s,t}\left(f_{s,t}^{\operatorname{d}}+f_t^{\operatorname{t}}\right) - \sum_{r\in\Rst} z_{s,r,t}\left(\fd_{s,r}+\ft_r+\fd_{r,t}-\fd_{s,t}\right) + \left(1-x_{s,t}\right)}$} \nonumber \\
	& & & \text{for all } t\in\mathcal{T}, s\in\Nin(t) \tag{3.14} \\
	& x_{s,t}\in\{0,1\} & & \text{for all } (s,t)\in A \tag{3.20} \\
	& z_{s,r,t}\in\{0,1\} & & \text{for all } t\in\mathcal{T},s\in\Nin(t),r\in\Rst \tag{3.21} \\
	& e_s\in[0,1] & & \text{for all } s\in V\backslash\left\{d^{\operatorname{s}},d^{\operatorname{e}}\right\} \tag{3.22} \\
	& u_m\in\{0,1\} && \text{for all } m\in\mathcal{M}
\end{align}