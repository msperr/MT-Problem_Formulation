\chapter{Introduction}

\fxnote{Check customer and costumer}
\fxnote{Check linebreak for in-text formulas}
\fxnote{Lagrange Heuristic for customer-dependent splitting as outlook}
\fxnote{Remove KK16 in bibliography}

In this thesis, a heuristic for the routing of autonomous vehicles is refined. It is based on two master's theses \cite{Kaiser} and \cite{Knoll} which were created earlier on the same faculty. While \cite{Knoll} develops heuristic solution methods for a simplified model, \cite{Kaiser} examines a decomposition approach for solving the original model exactly. In contrast to the previous theses, an extended model is regarded here, where the possibility of multimodal transport is considered. First, the previously developed heuristic is extended to the possibility of multimodal transport. Then, this heuristic is integrated in the already existing optimal approach.

In commercial car sharing, a customer rents a car for a limited period of time. In the classic version, the customer gets the car on a fixed location and returns it to this location after usage. In contrast to this, free-floating car sharing allows the customer to pick any available car and to put it down somewhere in the operation area. The customer usually books the car beforehand, typically via a smartphone application. He pays a certain amount per minute of car usage. This method is obviously more customer-friendly since the customer has no effort in getting to and from the renting location. But this means significantly more effort for the car sharing supplier. He has to provide a comprehensive offer of available cars, such that there is always a car where the customer needs it. Further he is responsible for refueling and servicing the cars, wherever they are. customers may park their car where it suits them and simultaneously only rent a car if it is within a small walking distance to their current position. Therefore, the distribution of the cars heavily depends on the customer behavior. This might lead to an imbalance of supply and demand.

A possible solution for this is the usage of autonomous vehicles. Although they are not available on present day, this topic is highly researched. Autonomous vehicles may be available within the next ten to twenty years (cf. \cite{Hauser}). The obvious advantage of autonomous cars is that they do not need a driver. An autonomous car is able change to the position after satisfying a customer on its own. The car can drive to a refuel station or to a position where it is needed next. For the customer, this behavior is similar to calling a taxi. The car picks him up on his present location and takes him to his destination. The supplier profits since he does not need employees for refueling or relocating the cars.

Besides cars, it is often advantageous for the customer to use public transport. For suitable trips, the usage of public transport is often faster and particularly cheaper. It is further more efficient in a city with many cars. But with public transport, there also some bad facts arise. The next station may be too far away for walking or unfavorable changing times increase the total travel time. In these cases it is often a good idea to combine car sharing and public transport in one trip. The customer uses the car for driving to a station with a good connection and continues the trip by using public transport. The combination of different types of transport is called \enquote{multiple-leg}. While in the previous master's theses the number of legs was restricted to one, the here developed solution methods can cope with multiple legs.

The introduction of autonomous cars is a huge change in the service provided to the customer. Therefore, it is hard to predict the changing of the customer behavior. This estimation is not part of this thesis, the focus lies rather in the potential of autonomous cars. Therefore, actual renting data from present day are used. In order to model realistic alternatives for the customer the data are slightly modified, for example by splitting existing trips into a car trip and a public transport trip. The goal is to find an optimal fleet size for the car sharing provider. For finding the optimal fleet size, the number of needed cars has to be traded off against the total driven distance. And these costs have to be considered while providing a good service to the customer. The determined fleet size can be compared to the actual fleet size in order to estimate the cost saving with autonomous cars.

The setting for the underlying theses was the following: Each customer has a set of alternative trips, where one trip each has to be fulfilled in the solution. Each trip is a single-leg trip, \ie the trip can either be a car or a public transport trip. \cite{Kaiser} provides an approach to solve this problem exactly via column generation. \cite{Knoll} provides a fast heuristic via a time-dependent splitting of the trips. This heuristic requires the further restriction that each customer has only one alternative, \ie each trip has to be fulfilled. In contrast, this thesis provides solution methods without both of these restrictions. The heuristic developed in \cite{Knoll} is extended in order to cope with multiple-leg multimodal transport. The goal of this heuristic is to determine a good solution fast. Afterwards, the extended heuristic is integrated in the optimal framework developed in \cite{Kaiser}, in order to find an optimal solution for the problem.