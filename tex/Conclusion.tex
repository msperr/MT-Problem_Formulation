\chapter{Conclusion}
\label{ch:conclusion}

In this final chapter, we give a summary of the problem setting and the developed solution methods for computing a heuristical and an optimal solution. We mention the crucial test results and give an overview of possible extensions for future work.

We regard the optimal integration of autonomous cars in car sharing. We assume a set of customers with certain travel requests which are known to us in advance. The customers are satisfied by multimodal transport, \ie they use alternative routes that consist of car and public transport trips. The autonomous vehicles have a certain fuel range and the possibility to refuel. We aim a schedule of the vehicles such that for each customer one alternative is fulfilled and the fuel range is considered. This problem is classified as a Vehicle Scheduling Problem with resource constraints and cover constraints and is $\mathcal{NP}$-hard. The work is based on two previous master theses. \cite{Kaiser} and \cite{Knoll} provide a heuristic and an optimal approach for a simplified problem. Our task was to adapt the heuristic and the optimal approach to the current problem setting and link both solution methods.

First of all, we develop an arc flow formulation for modeling the problem. It is based on a task graph with nodes for vehicles and trips and edges for feasible connections. Further the possibility to refuel is considered there. We create a mixed-integer linear program that defines a flow on the task graph satisfying the fuel constraints and the cover constraints. The goal is to minimize the total cost, consisting of costs for vehicles, trips, deadhead trips and routes. This formulation is assumed from \cite{Kaiser} and \cite{Knoll}, only the cover constraints are adapted.

In order to create a heuristic, we split the whole instance at some time points. Then we solve the emerging partial instances successively, beginning with the last one. In each partial instance, we connect the duties to the duties of the previously solved partial instance. This provides a feasible schedule that fulfills the fuel constraints. For satisfying the cover constraints, we examine two approaches: In the customer-dependent splitting we enforce all trips of the same customer to be in the same partial instance and apply the cover constraints separately there. Unfortunately, this splitting cuts off feasible solutions. In the time-dependent splitting we may have trips of one customer in different partial instances. We select the routes by using a cost estimation instead of full information. Further, we use the Iterative Heuristic in order to improve a heuristic solution. Customers with potentially bad routes choices are identified and their route choices are reviewed iteratively. Evaluating these heuristics on several test instances, we learn that the heuristics with both splitting types lead to very good results for large instance sizes. The iterative approach further improves an existing solution in reasonable time.

In order to obtain an optimal solution for the considered problem, we adapt a path flow formulation developed by \cite{Kaiser}. Here, the set of feasible duties for each vehicle is regarded. We apply Dantzig-Wolfe-Decomposition, where only the cover constraints are left in the master problem and the creation of feasible vehicle duties is referred to the subproblems. The subproblems are equivalent to a Shortest Path Problem with Resource Constraints where we introduce the reduced cost, fuel, duty length and fulfilled routes as resources. We use a label-setting algorithm in order to find a vehicle duty with negative reduced cost. With inverting the subproblem and searching paths from the end to the start, we solve all subproblems in just one execution. Further, we use both a heuristic and an optimal algorithm for generating new columns. Having a solution of the relaxed master problem, we enforce integrality by a Branch-and-Bound procedure. For computing an initial solution for the column generation process, we use the Successive Heuristic with a time-dependent splitting. Applying the algorithm on test instances, we see that only very small instances can be solved to optimality.

\paragraph{Open Topics} \parfill

Finally, we give an overview of conceivable further topics that are related to this thesis. This includes possible model extensions as well as further solution methods. 

In order to make the model more realistic, it should be applicable on several days in a row. In the current setting, a vehicle starts with some initial fuel state, but it may be completely discharged at the end of a duty. In a more realistic model, a vehicle has the possibility to drive to a refuel point at the end of a duty and its fuel level at the end of the instance amounts at least the initial fuel level. Moreover in the current setting, the vehicles may drive to each refuel point all the time. To be more realistic, each refuel point has some capacity as there is only a restricted number of power sockets. Thus, only a certain number of vehicles can be charged at some refuel point simultaneously. The creation of the multimodal routes with more than one car trip has potential for improvement. The approach that is used here is only a transitional solution in order to generate suitable test instances. The generation of multimodal routes that are optimal in some sense stays open for further investigation.

An additional heuristic for the problem without cover constraints is based on Lagrangean Relaxation (cf.~\cite[Chapter~9]{Knoll}). In this approach, the constraints connecting the partial instances are relaxed. In the customer-dependent splitting, all cover constraints occur in the partial instances. Thus, relaxing the constraints between the partial instances is similar to the problem setting without cover constraints. Thus, applying the Lagrangean Relaxation in combination with a customer-dependent splitting might be worth trying.

In the optimal approach, there are further possibilities that potentially improve the computation time. We have examined the route resources and observed that their direct application is not applicable as this yields too many combinations. On the other hand, these resources are not necessary for all routes and may be dropped after all trips of the respective customer are treated. If a meaningful usage of the route resources improves the algorithm, stays open for further research. Additionally, we can think of other methods to link the heuristics and the optimal approach. As the optimal approach performs better on smaller time windows (cf.~\cite[Sec.~10.2]{Kaiser}), applying the optimal approach only at the peak and continue the solution with heuristical methods might be a promising approach.

Finally, we consider the routing of autonomous vehicles only as a deterministic model. This means, all travel requests for a complete day are known in advance. In a more realistic setting, the travel requests occur only shortly before the desired start time. In the resulting online model, besides a small number of vehicles possibly short waiting times for the customers can be aimed.