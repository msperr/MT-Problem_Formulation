\section{Ideas}

\subsection{LP Approach}

We consider the possibility of multimodal transport. The following holds:

\begin{itemize}
	\item{Each costumer has a set of alternative multimodal routes; exactly one of them must be fulfilled.}
	\item{A multimodal route consists of a sequence of trips; each of them can be a car trip or a public transport trip.}
\end{itemize}

The routes are given in advance. The car trips are adjusted in such a way, that they fit to the public transport routes (in location and time). It is not possible to model each route as a trip, because then the car availabilities are not considered. 

We model a flow of the cars. The public transport trips are only constraints for this flow. From the fulfilled routes, we can easily determine the corresponding public transport trips and maybe add their costs to the objective function.

\paragraph{MMILP} \parfill

Let $\mathcal{T}$ be the set of car trips, $\mathcal{M}$ the set of multimodal routes, $\mathcal{C}$ the set of costumers. Then, $C:\mathcal{M}\to\mathcal{C}$ maps the routes to costumers and $M:\mathcal{T}\to\mathcal{M}$ maps the trips to routes. 

We build on the $\eqref{eq:MILP}$. The constraint $\eqref{eq:costumer_prev}$ is modified to

\begin{align}
\label{eq:costumer}
	\sum_{t\in\left(M\circ C\right)^{-1}(c)} \sum_{s\in\Nin(t)} x_{s,t} \geq 1 & & \text{for all } c\in\mathcal{C}.
\end{align}

We have $\geq 1$ instead of $= 1$ because the route can have more than one trip. Due to minimization, not more than one route will be selected. To guarantee, that each trip of a selected route is fulfilled, we insert

\begin{align}
\label{eq:route}
	\sum_{s\in\Nin(t)} x_{s,t} \geq \sum_{s\in\Nin(t')} x_{s,t'} & & \text{for all } t\in\mathcal{T}, t'\in M^{-1}(t)\backslash\{t\}.
\end{align}

\paragraph{Considerations}

\begin{enumerate}
	\item{Is it sufficient to only allow more than one leg for multimodal transport?}
	\item{Are there more effects on the model? No refueling at public transport}
	\item{What are the costs for public transport? No vehicle costs, only trip costs}
	\item{How are the data for public transport created?}
	\item{Is there a connection between the heuristic (Knoll) and the optimal approach (Kaiser)? Branch and Bound with result of the heuristic as a starting point}
	\item{Does the costumer prefer a route? Less total time, less costs, fewer changes}
\end{enumerate}

%---------------------------------------------------------------------------------------------------------------------------------------

\subsection{Heuristic}

This is an extension to the Successive Heuristic (Knoll, cap. 10). This heuristic further allows alternative routes for costumers. Each costumer has a set of alternative multimodal routes, consisting of a sequence of trips. For each costumer, exactly one route has to be fulfilled; if a route is fulfilled, every trip of this route is fulfilled.

\paragraph{First Approach} \parfill

In contrast to the previous heuristic, not the trips but the costumer requests are split. This means, all trips of a route and all routes of a costumer are in the same splitting. How the splitting is performed (e.g., first start time, last end time) is not specified here. For each splitting, we apply (MMILP) to find an optimal solution and connect the solutions as before. 

Our goal is the model equivalence between the heuristic and the (MMILP). For this, we consider the following example:

\begin{example}

Let $t_1$, $t_2$, $t_3$ with $t_1\to t_2\to t_3$ be trips with the following properties: 

\begin{table}[ht]
	\centering
	\begin{tabular}{c|cccc}
		Trip & Start & End & Route & Costumer \\
		\hline
		$t_1$ & 8:00 & 8:15 & $M_1$ & $C_1$ \\
		$t_2$ & 8:30 & 8:45 & $M_2$ & $C_2$ \\
		$t_3$ & 9:00 & 9:15 & $M_1$ & $C_1$ \\
	\end{tabular}
	\caption{Trips}
\end{table}

In this case, costumer $C_1$ uses public transport between 8:15 and 9:00. The duty $\left(t_1,t_2,t_3\right)$ is a feasible result of the (MMILP).

\end{example}

We distinguish between 3 cases:

\begin{enumerate}
	\item{$t_1,t_2,t_3$ are in the same splitting \\
		$\Rightarrow \left(t_1,t_2,t_3\right)$ is feasible since (MMILP)}
	\item{$t_1,t_3$ are in the same splitting, $t_2$ is before or after this splitting \\
		$\Rightarrow \left(t_1,t_3\right)$ and $\left(t_2\right)$ are not connectable}
	\item{$t_1,t_3$ are not in the same splitting \\
		$\Rightarrow$ this is not possible according to the heuristic}
\end{enumerate}

\begin{question}

Can we guarantee that case 2 cannot happen? Can we make restrictions to the splitting length, route extensions, trips lengths or choice of splitting, such that case 2 cannot happen? 

$\Rightarrow$ No. The splitting point can be arbitrarily bad.

\end{question}

\paragraph{Improved Approach} \parfill

With this example we have seen, that it has to be possible to split the trips of a costumer. If these trips are in different splittings, we are not able to maintain feasibility of the whole problem. Therefore, we need a new approach: 

We create the splittings as in the Successive Heuristic. Each trip is assigned to a splitting according to its start time. If one trip is chosen (in the earlier processed splitting), then the other trips of this route are fixed in their (later processed) splittings. This means, these trips have to be fulfilled in the later splittings.

\begin{example}

We have only one costumer $C_1$ with two alternative multimodal routes $M_1, M_2$. For this holds:

\begin{align*}
	M_1=\left(t_1,t_2\right) \quad M_2=\left(t_3\right) & & \mathcal{T}_1=\left\{t_1\right\} \quad \mathcal{T}_2=\left\{t_2,t_3\right\}
\end{align*}

We assume that $\mathcal{T}_1$ is processed first. If we set the costumer constraints $\eqref{eq:costumer}$ in $\mathcal{T}_1$, then the heuristic is forced to use $M_1$. Otherwise, the heuristic does not choose $t_1$ in $\mathcal{T}_1$ due to optimality and therefore is forced to use $M_2$.

\end{example}

To avoid this problem, we insert dummy trips when this is necessary. This means, we insert

\begin{align*}
	t^{\operatorname{d}}_{i,m} & & \text{for all } c\in\mathcal{C}, m\in C^{-1}(c), i \in[n]
\end{align*}

if there exist $s\in(M\circ C)^{-1}(c)$ with $s\in\mathcal{T}_i$ and there is no $s'\in M^{-1}(m)$ with $s'\in\mathcal{T}_i$. 

The dummy trips do not affect the model except for the costumer constraint $\eqref{eq:costumer}$ and in the objective function. We define $\mathcal{T}^{\operatorname{d}}_{i,c}$ as the set of dummy nodes for splitting $i\in[n]$ and costumer $c\in\mathcal{C}$. Then, the modified costumer constraints in splitting $i$ are

\begin{align}
\label{eq:costumer_new}
	\sum_{t\in(M\circ C)^{-1}(c)} \sum_{s\in\Nin(t)} x_{s,t} + \sum_{t\in\mathcal{T}^{\operatorname{d}}_{i,c}} y_t \geq 1 & & \text{for all } c\in\mathcal{C}
\end{align}

Using this approach, we have to decide already in splitting 1 which multimodal route is chosen for costumer $C_1$. This is difficult because we do not know which costs arise with using $M_1$ or $M_2$ in advance.

\begin{question}

How can we estimate the costs for the multimodal routes?

\end{question}

We can introduce some multimodal route costs, by which we decide at the beginning of a splitting, which route is chosen for the costumer. There we have to consider not only the sum of the trip costs but also factors like the fuel states or the location of the cars (to reduce the deadhead costs). The route costs can be introduced similar to $w^{\operatorname{heur}}$ (Kaiser, Knoll, cap. 4).

\paragraph{Feasibility} \parfill

The heuristic has to provide only feasible solutions for the main problem. Furthermore, each feasible solution of the main problem should be a feasible result in the heuristic. 

Except for the choice of trips, the subproblems are solved and connected in a feasible way with the Successive Heuristic. The dummy nodes only occur in $\eqref{eq:costumer_new}$ and in the objective, the trip choice only in $\eqref{eq:route}$ and $\eqref{eq:costumer_new}$. These constraints are maintained as follows: The route is chosen for the costumer definitively in the first splitting where the costumer occurs (either by a trip or a dummy trip). Therefore, each costumer has a route $\eqref{eq:costumer_new}$. Then, all other trips of this costumer are fixed in the following splittings $\eqref{eq:route}$. Hence, the heuristic provides a feasible solution. 

Consider a feasible overall solution. The trips that have to be used in this solution are already chosen in a feasible way. Then, the heuristic is the same as the Successive Heuristic (Knoll, cap. 10).

\paragraph{Considerations}

\begin{enumerate}
	\item{How long are the routes? Important for the choice of the splitting lengths}
	\item{Are all routes of a costumer in a similar time window? Restrict number of possible splittings for costumers}
\end{enumerate}