\chapter{Introduction}
\label{ch:introduction}

\fxnote{Check customer and costumer}
\fxnote{Check linebreak for in-text formulas}
\fxnote{Lagrange Heuristic for customer-dependent splitting as outlook}
\fxnote{Remove KK16 in bibliography}
\fxnote{Minimize duty cost in heuristical solution}
\fxnote{CMILPi: FuelLinkage, wo. ds}

In this thesis, we extend an already developed heuristic for the routing of autonomous vehicles. The thesis is based on two master's theses \cite{Kaiser} and \cite{Knoll} which were created earlier at the same department. The routing can be used for introducing autonomous vehicles into car sharing.

In commercial car sharing, a customer rents a car for a limited period of time. In the classic version, the customer gets the car on a fixed location and returns it to this location after usage. In contrast to this, free-floating car sharing allows the customer to pick up any car where this is available and to park it somewhere in the operation area. The customer usually books the car beforehand, typically via a smartphone application. He pays a certain amount per minute of car usage. This method is obviously more customer-friendly since the customer has no effort in getting to and from the renting location. But this means significantly more effort for the car sharing supplier. He has to provide a comprehensive offer of available cars, such that there is always a car where the customer needs it. Further he is responsible for refueling and servicing the cars, wherever they are. Customers may park their car where it suits them and simultaneously only rent a car if it is within a small walking distance to their current position. Therefore, the distribution of the cars heavily depends on the customer behavior. This might lead to an imbalance of supply and demand.

A possible solution for this is the usage of autonomous vehicles. Although they are not available on present day, this topic is highly researched. Autonomous vehicles may be available within the next ten to twenty years (cf. \cite{Hauser}). The obvious advantage of autonomous cars is that they do not need a driver. An autonomous car is able to change the position after satisfying a customer on its own. The car can drive to a refuel station or to a position where it is needed next. For the customer, this behavior is similar to calling a taxi. The car picks him up on his present location and takes him to his destination. The supplier profits since he does not need employees for refueling or relocating the cars.

Besides cars, it is often advantageous for the customer to use public transport. For suitable trips, the usage of public transport is often faster and particularly cheaper. It is further more efficient in a city with many cars. On the other hand, the usage of public transport yields some inconveniences for the customer. The next station may be too far away for walking or unfavorable changing times increase the total travel time. In these cases it is often a good idea to combine car sharing and public transport in one journey. The customer uses the car for driving to a station with a good connection and continues with using public transport. Therefore, the complete journey is divided into parts which we call \enquote{leg}. In each leg, the customer drives a certain distance without changing means of transport. The combination of different types of transport in one journey is called \enquote{multi-leg}.

The introduction of autonomous cars bears great potential for improvement for the car sharing supply. It involves huge changes for the maintenance of the vehicle fleet. The fleet means the number of vehicles that a car sharing supplier provides in a certain operation area. In order to estimate the profitability of introducing autonomous cars, the supplier is highly interested in the size of a vehicle fleet that is sufficient to maintain the car sharing supply. Therefore, we aim to find an optimal fleet size for the car sharing provider in this thesis. For this a small number of vehicles and a small driven distance in total is aspired, while still a good service shall be provided to the customer. Since the introduction of autonomous vehicles involves a great alteration in the service provided to the customer, it is hard to predict the change of the customer behavior. Therefore, we use current renting data in order to model the customer demand. With this we can estimate the improvement of autonomous vehicles compared to the current situation.

As mentioned at the beginning, the problem setting of this thesis is based on two previous theses. \cite{Kaiser} and \cite{Knoll} examine simplified versions of autonomous vehicle routing. The routes are restricted to have only a single leg there. While \cite{Knoll} provide fast heuristical solution methods via a time-dependent splitting of the trips, \cite{Kaiser} develop an approach to solve this problem to optimality via a branch-and-price process. In this thesis, the heuristical methods are extended in order to cope with multi-leg routes. The goal is the determination of a good initial solution. With adapting the optimal approach and using the computed initial solution, an optimal solution for autonomous vehicle routing is determined.