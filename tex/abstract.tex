\cleardoublepage

\selectlanguage{english}
\section*{Abstract}

In this thesis, we examine the potential impact of autonomous vehicles on car sharing. We examine the required fleet size and the arising cost in a deterministic model, if customers use multimodal transport. We create a mixed-integer linear program that extends the Vehicle Scheduling Problem considering fuel constraints and multimodal transport. We decompose the problem into smaller instances and solve them successively to gain a heuristical solution. For this, we examine two splitting approaches. Further, we create an improvement heuristic. Additionally, we apply Dantzig-Wolfe decomposition and solve the problem to optimality by a Branch-and-Price procedure. We investigate methods for solving the subproblems efficiently and create problem-dependent branching rules. This thesis concludes with an evaluation of the presented methods.

\selectlanguage{ngerman}
\section*{Zusammenfassung}

In dieser Masterarbeit beschäftigen wir uns mit dem Potential, das Autonomes Fahren auf das Car Sharing hat. Wir untersuchen die erforderliche Anzahl an Fahrzeugen und die auftretenden Kosten, wenn die Kunden multimodalen Transport benutzen. Aufbauend auf dem Vehicle Scheduling Problem erstellen wir ein gemischt-ganzzahliges lineares Optimierungsproblem, das die begrenzte Reichweite und den multimodalen Transport berücksichtigt. Wir zerlegen das Problem in kleinere Instanzen und lösen diese hintereinander, um eine heuristische Lösung zu erreichen. Dafür untersuchen wir zwei Zerlegungsansätze. Zusätzlich erstellen wir eine Heuristik, die eine vorhandene Lösung verbessert. Außerdem wenden wir eine Dantzig-Wolfe-Zerlegung an und lösen das Problem optimal mit einem Branch-and-Price-Verfahren. Wir analysieren Methoden um die Teilprobleme effizient zu lösen und erstellen Branching-Regeln, die auf das Problem aufbauen. Den Abschluss der Arbeit bildet eine Auswertung der entwickelten Methoden.

\selectlanguage{english}