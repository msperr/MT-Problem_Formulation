\chapter{Instance Creation}
\label{ch:instance_creation}

\section{Route Creation}

We are not given the set of routes $\mathcal{M}$ in advance. For each customer $c\in\mathcal{C}$, we have start and end location $\pstart_c, \pend_c$ and a start and end time $\zstarto_c, \zendo_c$. All the trips of the customer lie in this interval, i.e.

\begin{align*}
	\zstarto_c \leq \zstart_m && \zend_m \leq \zendo_c && \text{for all } m\in C^{-1}(c).
\end{align*}

\paragraph{Basic Restrictions} \parfill

To simplify the creation of the routes, we make some assumptions. For every route $m\in\mathcal{M}$ holds:

\begin{itemize}
	\item{There are not two car trips in a row.}
	\item{There is no car trip between two public transport trips.}
	\item{The number of public transport trips is restricted. Usually, one can reach every station with at most two changes.}
	\item{We define a walking distance $d^{\operatorname{walk}}$. If the distance between the start position and the first station or between the last station and the end position, there is no car trip necessary.}
\end{itemize}

We assume that we have some oracle that provides the set of feasible public transport routes for customer $c\in\mathcal{C}$:

\begin{align*}
	M_c & = && \left\{\left(s_1,z_1,s_2,z_2\right)|s_1,s_2\in\mathcal{S}, \zstarto_c\leq t_1 < t_2 \leq \zendo_c, \text{ there is a public} \right. \\
	&&& \left.\vphantom{\zstarto}\text{transport route from $s_1$ to $s_2$ with start time $z_1$ and end time $z_2$}\right\}
\end{align*}

The fact, whether the customer changes during his usage of public transport, has no effect on the model. Thus, we can consider each element in $M_c$ as a public transport trip.

\paragraph{Route Creation} \parfill

We create the set of multimodal routes $\mathcal{M}$. For this, we set a car trip before and after each public transport trip in order to bring the customer from his start to his destination, except when it is possible to walk the distance. We also have to consider the given time restrictions. Further, we create the pure car trips. How the set $\mathcal{M}$ is created in detail, is described in \Cref{alg:route_creation}.

Until now, we do not consider any changing times between a car trip and a public transport trip.

Further, we assume that the given customer start and and times are feasible, i.e. $\zstarto_c+t_{\pstart_c,\pend_c}\leq\zendo_c$ for all $c\in\mathcal{C}$.

\begin{algorithm}
	\SetAlgoLined
	\KwIn{customer set $\mathcal{C}$; $\pstart_c, \pend_c, \zstarto_c, \zendo_c, M_c$ for all $c\in\mathcal{C}$}
	\KwOut{set of routes $\mathcal{M}$, set of trips $\Tcar,\Tpub$}
	$\Tcar\gets\emptyset$\;
	$\Tpub\gets\emptyset$\;
	$\mathcal{M}\gets\emptyset$\;
	\ForEach{$c\in\mathcal{C}$}{
		\ForEach{$\left(s_1,z_1,s_2,z_2\right)\in M_c$}{
			create public transport trip $t$\;
			$\pstart_t\gets s_1,\pend_t\gets s_2,\zstart_t\gets z_1,\zend_t\gets z_2$\;
			create car trips $t_1,t_2$\;
			$\pstart_{t_1}\gets\pstart_c,\pend_{t_1}\gets s_1,\zstart_{t_1}\gets z_1 - t_{\pstart_c,s_1},\zend_{t_1}\gets z_1$\;
			$\pstart_{t_2}\gets s_2,\pend_{t_2}\gets\pend_c,\zstart_{t_2}\gets z_2,\zend_{t_2}\gets z_2 + t_{s_2,\pend_c}$\;
			\If{$\zstarto_c\leq\zstart_{t_1} \land \zend_{t_2}\leq \zendo_c$}{
				create multimodal route $m$\;
				$\Tpub\gets\Tpub\cup \{t$\}\;
				\lIf{$d_{\pstart_c,s_1}\geq d^{\operatorname{walk}}$}{$m\gets\left(t_1,t\right)$; $\Tcar\gets\Tcar\cup\left\{t_1\right\}$}
				\lElse{$m\gets\left(t\right)$}
				\lIf{$d_{s_2,\pend_c}\geq d^{\operatorname{walk}}$}{append $t_2$ to $m$; $\Tcar\gets\Tcar\cup\left\{t_2\right\}$}
				$C(m)\gets c$\;
				$\mathcal{M}\gets\mathcal{M}\cup\left\{m\right\}$\;
				}
		}
		create car trips $t_1,t_2$\;
		$\pstart_{t_1}\gets\pstart_c,\pend_{t_1}\gets\pend_c,\zstart_{t_1}\gets\zstarto_c,\zend_{t_1}\gets\zstarto_c + t_{\pstart_c,\pend_c}$\;
		$\pstart_{t_2}\gets\pstart_c,\pend_{t_2}\gets\pend_c,\zstart_{t_2}\gets\zendo_c - t_{\pstart_c,\pend_c},\zend_{t_2}\gets\zendo_c$\;
		create multimodal routes $m_1,m_2$\;
		$m_1\gets\left(t_1\right), m_2\gets\left(t_2\right)$\;
		$\Tcar\gets\Tcar\cup\left\{t_1,t_2\right\}, \mathcal{M}\gets\mathcal{M}\cup\left\{m_1,m_2\right\}$\;
	}
	\Return{$\mathcal{M},\Tcar,\Tpub$}
	\caption{Creation of the routes \label{alg:route_creation}}
\end{algorithm}

\paragraph{Further Restrictions} \parfill

If the routes are created as described in \Cref{alg:route_creation}, there are routes using every available station as long as it is feasible. Most of these routes are obviously bad for the customer since they cause a big detour. What is more, a large number of routes enlarge the problem size and leads to  a bad performance for solving it. Therefore, we try to restrict the set of alternatives to a reasonable size.

\begin{example}

Let $\mathcal{S}=\left\{s_1,\dots,s_n\right\}$ with a single public transport ride serving all stations. Let $\mathcal{C}=\left\{c_1,c_2\right\}$ with $\pend_{c_1}=s_n$ and $\pstart_{c_2}=s_k$ for a certain $k\in[n-1]$. The alternative routes are

\begin{align*}
	\mathcal{M} = \underbrace{\left\{\left(\left(\pstart_{c_1},s_i\right),\left(s_i,s_n\right)\right)|i\in[n-1]\right\}}_{\text{for }c_1}\cup\underbrace{\left\{\left(s_k,\pend_{c_2}\right)\right\}}_{\text{for }c_2}
\end{align*}

with $\left(\pstart_{c_1},s_k\right)\prec\left(s_k,\pend_{c_2}\right)$ and $\left(\pstart_{c_1},s_i\right)\not\prec\left(s_k,\pend_{c_2}\right)$ for all $i\in[n]\backslash\{k\}$. 

We get the only solution, where only one car is needed, when $c_1$ drives to $s_k$, wherever the station $s_k$ is. Every route of $c_1$ can be the optimal route, considering the other customers. Therefore, an exact reduction of $\mathcal{M}$ is not possible without the the risk of cutting off the optimal solution.
	
\end{example}

It is not practicable to consider all possible multimodal routes due to computation reasons. But it is also not possible to reduce the number of routes without risking to lose the optimal solution. Hence, we try to make reasonable restrictions which keep the problem size small.

\paragraph{Pareto Optimality} \parfill

The idea is to choose only Pareto optimal multimodal routes (cf. Kaiser/Knoll, cap. 3.2.2) in order to determine good routes.

\begin{definition}[Pareto optimality]

Let $V\subset\mathbb{R}^n$.

\begin{enumerate}
	\item{The partial order $\leq$ on $\mathbb{R}^n$ is given by
		\begin{align*}
			v\leq w && :\Leftrightarrow && v_i\leq w_i && \forall i\in[n] && \text{for all }v,w\in\mathbb{R}^n
		\end{align*}}
	\item{An element $w\in V$ is Pareto optimal in $V$ if it is minimal with respect to $\leq$ in $V$, i.e.
		\begin{align*}
			v\leq w && \Rightarrow && v=w && \text{for all } v\in V
		\end{align*}}
	\item{The Pareto frontier of $V$ with respect to $\leq$ is the set of Pareto optimal elements in $V$, i.e.
		\begin{align*}
			\operatorname{min}_{\leq} V := \left\{w\in V|\forall v\in V: v\leq w \Rightarrow v=w\right\}
		\end{align*}}
\end{enumerate}

\end{definition}

Let $m\in\mathcal{M}$ be a multimodal route. We define

\begin{align*}
	\varphi:\mathcal{M}\to\mathbb{R}^5 && m\mapsto\left(
	\begin{array}{c}
		\croute + \sum_{t\in m\cap\Tcar}\ct_t \\
		\croute \\
		|\Tcar\cap\{t\in m\}|\\
		\sum_{t\in m\cap\Tcar} \zend_t - \zstart_t \\
		\sum_{t\in m\cap\Tcar} \ft_t
	\end{array} \right)
\end{align*}

The function $\varphi$ grades a route to their costs, their route costs, the number of cars needed, the time of a car needed and the fuel consumption.

From now on, we will use the Pareto frontier of $\varphi\left(\mathcal{M}\right)$ as a restricted route set:

\begin{align}
	\hat{\mathcal{M}} := min_{\leq} \varphi\left(\mathcal{M}\right)
\end{align}
